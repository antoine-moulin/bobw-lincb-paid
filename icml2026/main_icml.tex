%%%%%%%% ICML 2026 EXAMPLE LATEX SUBMISSION FILE %%%%%%%%%%%%%%%%%

\documentclass{article}

% Recommended, but optional, packages for figures and better typesetting:
\usepackage{microtype}
\usepackage{graphicx}
\usepackage{subcaption}
\usepackage{booktabs} % for professional tables

% hyperref makes hyperlinks in the resulting PDF.
% If your build breaks (sometimes temporarily if a hyperlink spans a page)
% please comment out the following usepackage line and replace
% \usepackage{icml2026} with \usepackage[nohyperref]{icml2026} above.
\usepackage{ninecolors}
\usepackage[colorlinks=true,
            linkcolor=blue4,
            filecolor=blue4,
            urlcolor=blue4,
            citecolor=blue4]{hyperref}

% Attempt to make hyperref and algorithmic work together better:
\newcommand{\theHalgorithm}{\arabic{algorithm}}

% Use the following line for the initial blind version submitted for review:
\usepackage{icml2026}

% For preprint, use
% \usepackage[preprint]{icml2026}

% If accepted, instead use the following line for the camera-ready submission:
% \usepackage[accepted]{icml2026}

\usepackage{amsmath}
\usepackage{amssymb}
\usepackage{mathtools}
\usepackage{amsthm}
\usepackage{thmtools}
\usepackage{thm-restate}

% special macros
% paper specific
\newcommand{\XMAX}{X_{\mathrm{max}}}
\newcommand{\THETAMAX}{\Theta_{\mathrm{max}}}

% comments
\usepackage{ninecolors}
\newcommand{\antoine}[1]{%
    \ifmmode
    \text{\textcolor{green6}{[Antoine: #1]}}
    \else
    \textcolor{green6}{[Antoine: #1]}
    \fi
}

\newcommand{\tmpassumption}[1]{%
    \ifmmode
    \text{\textcolor{orange}{[Assumption: #1]}}
    \else
    \textcolor{orange}{[Assumption: #1]}
    \fi
}

% common abbreviations
\usepackage{xspace}
\makeatletter
\DeclareRobustCommand\onedot{\futurelet\@let@token\@onedot}
\def\@onedot{\ifx\@let@token.\else.\null\fi\xspace}

\def\cf{\textit{c.f}\onedot}
\def\eg{\textit{e.g}\onedot}
\def\etal{\textit{et al}\onedot}
\def\etc{\textit{etc}\onedot}
\def\ie{\textit{i.e}\onedot}
\def\iid{i.i.d\onedot}
\def\vs{\textit{vs}\onedot}
\def\wrt{w.r.t\onedot}
\makeatother

% surrounding symbols
\DeclarePairedDelimiter{\spr}{(}{)}
\DeclarePairedDelimiter{\sbr}{[}{]}
\DeclarePairedDelimiter{\sdbr}{[\![}{]\!]}
\DeclarePairedDelimiter{\scbr}{\{}{\}}
\DeclarePairedDelimiter{\inp}{\langle}{\rangle}
\DeclarePairedDelimiter{\norm}{\|}{\|}
\DeclarePairedDelimiter{\abs}{|}{|}

% basic algebra symbols
\newcommand{\sumin}{\sum_{i=1}^n}
\newcommand{\sumaK}{\sum_{a=1}^K}
\newcommand{\sumiK}{\sum_{i=1}^K}
\newcommand{\sumjK}{\sum_{j=1}^K}
\newcommand{\sumkK}{\sum_{k=1}^K}
\newcommand{\sumtT}{\sum_{t=1}^T}
\newcommand{\sumst}{\sum_{s=1}^t}
\newcommand{\sumtinfty}{\sum_{t=0}^\infty}
\newcommand{\sumsinfty}{\sum_{s=0}^\infty}

\newcommand*\diff{\mathop{}\!\mathrm{d}}
\usepackage{mathtools} % for \mathclap
%\newcommand{\given}{\mathrel{}\middle|\mathrel{}}
%\newcommand{\given}{\;\middle|\;}
\newcommand{\given}{\nonscript\mathpunct{}\middle|\nonscript\mathpunct{}}

\newcommand{\DKL}[2]{\mathcal{D}_{\mathsf{KL}}\spr*{#1,#2}}
\newcommand{\Dhel}[2]{\mathcal{D}_{\mathsf{H}}\spr*{#1,#2}}
\newcommand{\Dhels}[2]{\mathcal{D}_{\mathsf{H}}^2\spr*{#1,#2}}
\newcommand{{\transpose}}{^\mathsf{\scriptscriptstyle T}}

\newcommand{\wh}[1]{\widehat{#1}}
\newcommand{\wt}[1]{\widetilde{#1}}

\usepackage{mathtools}  % for '\mathrlap' command (necessary)
\newcommand{\custombar}[3]{%
    \mathrlap{\hspace{#2}\overline{\scalebox{#1}[1]{\phantom{\ensuremath{#3}}}}}\ensuremath{#3}
} % random example: \custombar{0.8}{.5pt}{y}

% blackboard bold letters
\newcommand{\bbC}{\mathbb{C}}
\newcommand{\bbE}{\mathbb{E}}
\newcommand{\bbF}{\mathbb{F}}
\newcommand{\bbI}{\mathbb{I}}
\newcommand{\bbN}{\mathbb{N}}
\newcommand{\bbP}{\mathbb{P}}
\newcommand{\bbQ}{\mathbb{Q}}
\newcommand{\bbR}{\mathbb{R}}
\newcommand{\bbRnnK}{\mathbb{R}_{\geq 0}^K}
\newcommand{\bbRpK}{\mathbb{R}_{>0}^K}
\newcommand{\bbV}{\mathbb{V}}
\newcommand{\bbZ}{\mathbb{Z}}

% bold letters
\newcommand{\bfone}{\mathbf{1}}
\newcommand{\bfe}{\mathbf{e}}
\newcommand{\bfu}{\mathbf{u}}
\newcommand{\bfv}{\mathbf{v}}
\newcommand{\bfw}{\mathbf{w}}
\newcommand{\bfx}{\mathbf{x}}
\newcommand{\bfy}{\mathbf{y}}
\newcommand{\bfyt}{\mathbf{y}_t}
\newcommand{\bfz}{\mathbf{z}}

% caligraphic letters
\newcommand{\cA}{\mathcal{A}}
\newcommand{\cB}{\mathcal{B}}
\newcommand{\cC}{\mathcal{C}}
\newcommand{\cD}{\mathcal{D}}
\newcommand{\cE}{\mathcal{E}}
\newcommand{\cF}{\mathcal{F}}
\newcommand{\cG}{\mathcal{G}}
\newcommand{\cH}{\mathcal{H}}
\newcommand{\cI}{\mathcal{I}}
\newcommand{\cJ}{\mathcal{J}}
\newcommand{\cK}{\mathcal{K}}
\newcommand{\cL}{\mathcal{L}}
\newcommand{\cM}{\mathcal{M}}
\newcommand{\cN}{\mathcal{N}}
\newcommand{\cO}{\mathcal{O}}
\newcommand{\cP}{\mathcal{P}}
\newcommand{\cQ}{\mathcal{Q}}
\newcommand{\cR}{\mathcal{R}}
\newcommand{\cS}{\mathcal{S}}
\newcommand{\cT}{\mathcal{T}}
\newcommand{\cU}{\mathcal{U}}
\newcommand{\cV}{\mathcal{V}}
\newcommand{\cX}{\mathcal{X}}
\newcommand{\cY}{\mathcal{Y}}
\newcommand{\cZ}{\mathcal{Z}}

% fraktur letters
\newcommand{\fkB}{\mathfrak{B}}
\newcommand{\fkR}{\mathfrak{R}}

% math operators
\DeclareMathOperator*{\argmax}{arg\,max}
\DeclareMathOperator*{\argmin}{arg\,min}
\DeclareMathOperator{\co}{co}
\DeclareMathOperator{\Diag}{Diag}  % diagonal matrices
\DeclareMathOperator{\diam}{diam}  % diameter
\DeclareMathOperator{\dom}{dom}  % domain
\DeclareMathOperator{\interior}{int}
\DeclareMathOperator{\Lip}{Lip} % lipschitz
\DeclareMathOperator{\scvx}{sc}  % strong convexity
\DeclareMathOperator{\sign}{sign}
\DeclareMathOperator{\optr}{tr}  % trace
\DeclareMathOperator{\softmax}{softmax}
\DeclareMathOperator{\conv}{conv}
\DeclareMathOperator{\LSE}{LSE}

% others
\newcommand{\simplex}{\Delta}
\newcommand{\piunif}{\pi_{\texttt{unif}}}
\newcommand{\reset}{reset\xspace}
\newcommand{\ptmin}{p_{t, \min}}
\newcommand{\biasmgr}{\texttt{bias}_{\texttt{MGR}}}

% maths stuff
\usepackage{bm}
\newcommand{\R}{\bbR}
\newcommand{\N}{\mathbb{N}}

% argmax and co
\newcommand{\aargmax}{\text{argmax}}
\newcommand{\aargmin}{\text{argmin}}

% probability
\newcommand{\bP}{\mathbb{P}}
\newcommand{\bE}{\bbE}
%\newcommand{\ind}{\mathds{1}}
\usepackage{dsfont}
\newcommand{\ind}{\mathds{1}}
%\newcommand{\mathbbm}[1]{\text{\usefont{U}{bbm}{m}{n}#1}}
\newcommand{\kl}{\mathrm{kl}}
\newcommand{\dm}{\mathrm{d}}

% bandit and algorithm specific notation 
\newcommand{\K}{\{1,\dots,K\}}
\newcommand{\SP}{\ensuremath{\mathrm{SP}}}
\newcommand{\UCB}{\mathrm{UCB}}

% algorithms
\newcommand{\RBSSDA}{\ensuremath{\mathrm{RB}\text{-}\mathrm{SDA}}}
\newcommand{\SSDA}{\ensuremath{\mathrm{SDA}}}
\newcommand{\SDA}{\ensuremath{\mathrm{SDA}}}
\newcommand{\SPSSDA}{\ensuremath{\mathrm{SP}\text{-}\mathrm{SDA}}}
\newcommand{\WRSSDA}{\ensuremath{\mathrm{WR}\text{-}\mathrm{SDA}}}
\newcommand{\LBSSDA}{\ensuremath{\mathrm{LB}\text{-}\mathrm{SDA}}}
\newcommand{\LDSSSDA}{\ensuremath{\mathrm{LDS}\text{-}\mathrm{SDA}}}
\newcommand{\sampler}{independent sampler}
\newcommand{\RB}{\ensuremath{\mathrm{RB}}}
\newcommand{\WR}{\ensuremath{\mathrm{WR}}}

\newcommand{\GNPTS}{DS}
\newcommand{\MCVTS}{$\alpha$-Multinomial-TS}
\newcommand{\CVTS}{$\alpha$-NPTS}

\newcommand{\CVAR}{\text{CVaR}}
\newcommand{\kinf}{\cK_{\inf}}
\newcommand{\kinfcv}{\cK_{\inf}^{\alpha,\cD}}
\newcommand{\kinfcvx}{\cK_{\inf}^{\alpha,\cX}}
\newcommand{\kinfcvxm}{\mathcal{G}_{\inf}^{\alpha,\cX}}
\newcommand{\KL}{\mathrm{KL}}

\newcommand{\Sm}{\cS_{m, \cX}^\alpha(c)}
\newcommand{\Smp}{\cU_{m, \cX}^\alpha(c)}
\newcommand{\Sa}{\cS_{\cX}^\alpha(c)}
\newcommand{\Sap}{\cU_{\cX}^\alpha(c)}

\newcommand{\MDir}{\hat{p}_{k,t}}
\newcommand{\betaDir}{\hat{\beta}_{k,t}}
\newcommand{\Dir}{\text{Dir}}

% redefinition of some commands 
\let\ln\relax
\DeclareMathOperator{\ln}{ln}
\let\epsilon\relax
\DeclareMathOperator{\epsilon}{\varepsilon}
\DeclareMathOperator*{\esssup}{ess\,sup}

\usepackage{tikz}
\usetikzlibrary{decorations.pathreplacing,calc}
\newcommand{\tikzmark}[1]{\tikz[overlay,remember picture] \node (#1) {};}

\newcommand*{\AddNote}[4]{%
	\begin{tikzpicture}[overlay, remember picture]
	\draw [decoration={brace,amplitude=0.5em},decorate,ultra thick,red]
	($(#3)!(#1.north)!($(#3)-(0,1)$)$) --  
	($(#3)!(#2.south)!($(#3)-(0,1)$)$)
	node [align=center, text width=2.5cm, pos=0.5, anchor=west] {#4};
	\end{tikzpicture}
}%

\newcommand{\sgone}{\mathrm{SG}_1}
\newcommand{\sgtwo}{\mathrm{SG}_2}
\newcommand{\sgthree}{\mathrm{SG}_2}
\newcommand{\gstar}{{G^*}}
\newcommand{\napier}{\mathrm{e}}
\newcommand{\E}{\mathrm{E}}
\newcommand{\e}{\mathrm{e}}

\newcommand{\since}[1]{\quad\left(\mbox{#1}\right)}
\newcommand{\rd}{\mathrm{d}}
\newcommand{\ep}{\varepsilon}
\newcommand{\nn}{\nonumber}
\newcommand{\n}{\nonumber}

\newcommand{\TS}{TS$^\star$}

\newcommand{\red}{\textcolor{red}}
\newcommand{\blue}{\textcolor{blue}}
\newcommand{\Loja}{Łojasiewicz}

\usepackage{soul}
\newcommand{\revrem}[1]{\textcolor{blue}{\st{#1}}}

\newcommand{\SGB}{\texttt{SGB}{ }}

\newcommand{\db}[1]{\textcolor{purple}{\#DB: #1 \#}}



% EJ:

% round brackets
\newcommand{\lrb}[1]{\left(#1\right)}
\newcommand{\brb}[1]{\bigl(#1\bigr)}
\newcommand{\Brb}[1]{\Bigl(#1\Bigr)}
\newcommand{\bbrb}[1]{\biggl(#1\biggr)}
\newcommand{\Bbrb}[1]{\Biggl(#1\Biggr)}
% square brackets
\newcommand{\lsb}[1]{\left[#1\right]}
\newcommand{\bsb}[1]{\bigl[#1\bigr]}
\newcommand{\Bsb}[1]{\Bigl[#1\Bigr]}
\newcommand{\bbsb}[1]{\biggl[#1\biggr]}
\newcommand{\Bbsb}[1]{\Biggl[#1\Biggr]}
% curly brackets
\newcommand{\lcb}[1]{\left\{#1\right\}}
\newcommand{\bcb}[1]{\bigl\{#1\bigr\}}
\newcommand{\Bcb}[1]{\Bigl\{#1\Bigr\}}
\newcommand{\bbcb}[1]{\biggl\{#1\biggr\}}
% ceiling
\newcommand{\lce}[1]{\left\lceil#1\right\rceil}
\newcommand{\bce}[1]{\bigl\lceil#1\bigr\rceil}
\newcommand{\Bce}[1]{\Bigl\lceil#1\Bigr\rceil}
\newcommand{\bbce}[1]{\biggl\lceil#1\biggr\rceil}
% floor
\newcommand{\lfl}[1]{\left\lfloor#1\right\rfloor}
\newcommand{\bfl}[1]{\bigl\lfloor#1\bigr\rfloor}
\newcommand{\Bfl}[1]{\Bigl\lfloor#1\Bigr\rfloor}
\newcommand{\bbfl}[1]{\biggl\lfloor#1\biggr\rfloor}
% abs
\newcommand{\labs}[1]{\left\lvert#1\right\rvert}
\newcommand{\babs}[1]{\bigl\lvert#1\bigr\rvert}
\newcommand{\Babs}[1]{\Bigl\lvert#1\Bigr\rvert}
\newcommand{\bbabs}[1]{\biggl\lvert#1\biggr\rvert}
% norm
\newcommand{\lno}[1]{\left\lVert#1\right\rVert}
\newcommand{\bno}[1]{\bigl\lVert#1\bigr\rVert}
\newcommand{\Bno}[1]{\Bigl\lVert#1\Bigr\rVert}
\newcommand{\bbno}[1]{\biggl\lVert#1\biggr\rVert}
% angle
\newcommand{\lan}[1]{\left\langle#1\right\rangle}
\newcommand{\ban}[1]{\bigl\langle#1\bigr\rangle}
\newcommand{\Ban}[1]{\Bigl\langle#1\Bigr\rangle}
\newcommand{\bban}[1]{\biggl\langle#1\biggr\rangle}

\newcommand{\deltamin}{\Delta_{\text{min}}}
\newcommand{\sube}{\subseteq}
\newcommand{\rchi}{\raisebox{2pt}{$\chi$}}
%%%

\usepackage{comment}
\usepackage{enumitem}

% \usepackage{algorithm} 
% \usepackage{algpseudocode}
% \usepackage{thmtools}
% \usepackage{thm-restate}

\usepackage[textsize=tiny]{todonotes}
\newcommand{\dodo}[2][]{\todo[color=blue!20,#1]{{\bf DB:} #2}}

\newcommand{\DB}[1]{\noindent{\textcolor{red}{\textbf{\#DB: }#1\#}}}

% if you use cleveref..
\usepackage[capitalize,noabbrev]{cleveref}

%%%%%%%%%%%%%%%%%%%%%%%%%%%%%%%%
% THEOREMS
%%%%%%%%%%%%%%%%%%%%%%%%%%%%%%%%
% \theoremstyle{plain}
% \newtheorem{theorem}{Theorem}[section]
% \newtheorem{proposition}[theorem]{Proposition}
% \newtheorem{Lem}{Lemma}
% \newtheorem{corollary}[theorem]{Corollary}
% \theoremstyle{definition}
% \newtheorem{definition}[theorem]{Definition}
% \newtheorem{assumption}[theorem]{Assumption}
% \theoremstyle{remark}
% \newtheorem{remark}[theorem]{Remark}

\theoremstyle{plain}
\declaretheorem[numberwithin=section,name=Theorem]{theorem}
\declaretheorem[sibling=theorem,name=Proposition]{proposition}
\declaretheorem[sibling=theorem,name=Lemma]{lemma}
\declaretheorem[sibling=theorem,name=Corollary]{corollary}
\theoremstyle{definition}
\declaretheorem[sibling=theorem,name=Definition]{definition}
\declaretheorem[sibling=theorem,name=Assumption]{assumption}
\theoremstyle{remark}
\declaretheorem[sibling=theorem,name=Remark]{remark}

% Todonotes is useful during development; simply uncomment the next line
%    and comment out the line below the next line to turn off comments
%\usepackage[disable,textsize=tiny]{todonotes}
\usepackage[textsize=tiny]{todonotes}


% The \icmltitle you define below is probably too long as a header.
% Therefore, a short form for the running title is supplied here:
\icmltitlerunning{Linear contextual bandits with paid observations}

\begin{document}

\twocolumn[
  \icmltitle{Best-of-Both Worlds for linear contextual bandits with paid observations}

  % It is OKAY to include author information, even for blind submissions: the
  % style file will automatically remove it for you unless you've provided
  % the [accepted] option to the icml2026 package.

  % List of affiliations: The first argument should be a (short) identifier you
  % will use later to specify author affiliations Academic affiliations
  % should list Department, University, City, Region, Country Industry
  % affiliations should list Company, City, Region, Country

  % You can specify symbols, otherwise they are numbered in order. Ideally, you
  % should not use this facility. Affiliations will be numbered in order of
  % appearance and this is the preferred way.
  \icmlsetsymbol{equal}{*}

  \begin{icmlauthorlist}
    \icmlauthor{Antoine Moulin}{upf}
    \icmlauthor{Nathan Boyer}{oxford}
    \icmlauthor{Patrick Rebeschini}{oxford}
    \icmlauthor{Dorian Baudry}{inria}
  \end{icmlauthorlist}

  \icmlaffiliation{upf}{Universitat Pompeu Fabra, Barcelona, Spain}
  \icmlaffiliation{oxford}{Department of Statistics, University of Oxford, Oxford, UK}
  \icmlaffiliation{inria}{Univ.\ Grenoble Alpes, Inria, CNRS, Grenoble INP, LIG, 38000 Grenoble, France}

  \icmlcorrespondingauthor{Dorian Baudry}{dorian.baudry@inria.fr}

  % You may provide any keywords that you find helpful for describing your
  % paper; these are used to populate the "keywords" metadata in the PDF but
  % will not be shown in the document
  \icmlkeywords{Linear contextual bandits, best-of-both-worlds, paid observations}

  \vskip 0.3in
]

% this must go after the closing bracket ] following \twocolumn[ ...

% This command actually creates the footnote in the first column listing the
% affiliations and the copyright notice. The command takes one argument, which
% is text to display at the start of the footnote. The \icmlEqualContribution
% command is standard text for equal contribution. Remove it (just {}) if you
% do not need this facility.

% Use ONE of the following lines. DO NOT remove the command.
% If you have no special notice, KEEP empty braces:
\printAffiliationsAndNotice{}  % no special notice (required even if empty)
% Or, if applicable, use the standard equal contribution text:
% \printAffiliationsAndNotice{\icmlEqualContribution}

\begin{abstract}
  We study the problem of linear contextual bandits with paid observations, where at each round the learner selects an action in order to minimize its loss in a given context, and can then decide to pay a fixed cost to observe the loss of any arm. Building on the Follow-the-Regularized-Leader framework with efficient estimators via Matrix Geometric Resampling, we introduce a computationally efficient Best-of-Both-Worlds (BOBW) algorithm for this problem. We show that it achieves the minimax-optimal regret of $\Theta(T^{2/3})$ in adversarial settings, while guaranteeing poly-logarithmic regret in (corrupted) stochastic regimes. Our approach builds on the framework from \cite{BOBWhardproblems} to design BOBW algorithms for ``hard problem'', using analysis techniques tailored for the setting that we consider. %, which may be of independent interest.
\end{abstract}

\input{sections/introduction.tex}
\section{Problem Definition} \label{sec::setting}

In this section we formalize the setting of \emph{linear bandits with paid observations}, and state the main assumptions used in the analysis presented in Section~\ref{sec::regret}.

\paragraph{Interaction protocol.}
The interaction between the learning agent and the environment has a total duration of $T \in \bbN$ time steps, where $T$ is unknown to the learner. Context vectors are drawn independently from a fixed distribution $\cD$ supported on a compact, full-dimensional subset $\cX \subseteq \bbR^d$. At each round $t$, the following steps occur:
%
\begin{enumerate}
    \item For each action $a \in \sbr*{K} \coloneqq \scbr*{1, \ldots, K}$, the environment selects a loss parameter $\theta_{t, a} \in \bbR^d$.
    \item A context $X_t \in \cX$ is drawn from $\cD$.
    \item The learner observes $X_t$, chooses an action $A_t \in \sbr*{K}$, and an observation set $O_t \subseteq \sbr*{K}$.
    \item The learner incurs loss $\ell_t \spr*{X_t,A_t} + c \abs*{O_t}$, where $\ell_t$ is a loss function that depends on the environment parameters $\spr*{\theta_{t, a}}_{a \in \sbr*{K}}$, $c > 0$ is the known unit cost of observation, and $\abs*{O_t}$ is the cardinality of the observation set. It then observes the losses $\scbr*{\ell_t \spr*{X_t, o} : o \in O_t}$.
\end{enumerate}

Following \citet{SeldinS14}, the learner may query multiple arms in each round, paying cost $c$ per queried arm. When $c = 0$, the learner is incentivized to query all arms, recovering the \emph{full-information} (or ``experts'') setting.

\paragraph{Assumptions.}
To enable algorithm design and analysis, we adopt standard assumptions from the linear contextual bandit literature \citep{BOBWlinear}:
%
\begin{enumerate}
    \item \textcolor{red}{For $X \sim \cD$, $\norm*{X}_2 \leq \XMAX$} almost surely.
    
    \item For any $t \in \sbr*{T}$, $a \in \sbr*{K}$, $\textcolor{red}{\norm*{\theta_{t, a}}_2 \leq \THETAMAX}$.
    
    \item For any $t \in \sbr*{T}$, $x \in \cX$, $a \in \sbr*{K}$, $\ell_t \spr*{x, a} \in \sbr*{-1, 1}$. \antoine{inconsistent with the noise model}
\end{enumerate}
%
We denote by $\Sigma = \bbE_{X \sim \cD} \sbr*{X X\transpose} \succ 0$ the covariance matrix of the context distribution, and by $\lambda_{\min} > 0$ its minimum non zero eigenvalue, assumed to be known to the learner. While the learner does not know $\cD$ in full, we assume access to independent samples from $\cD$ between rounds, for instance through a simulator.

We now define how the loss $\ell_t \spr*{x, a}$ is constructed in each of the regimes considered in this work, for a given step $t \in \sbr*{T}$, context $x \in \cX$ and arm $a \in \sbr*{K}$.

\paragraph{Adversarial regime.}
The loss satisfies $\ell_t \spr*{x, a} \coloneqq \inp*{x, \theta_{t, a}}$, where $\theta_{t, a}$ is chosen by an \emph{oblivious} adversary: the entire sequence $\spr*{\theta_{t, a}}_{t \in \sbr*{T}, a \in \sbr*{K}}$ can be arbitrary, but is fixed before the interaction starts.

\paragraph{Stochastic regime.}
The loss is defined by $\ell_t \spr*{x, a} \coloneqq \inp*{x, \theta_a} + \epsilon_{t, a}$ where $\theta_a$ is a fixed, unknown parameter for each arm $a$, and $\epsilon_{t, a}$ is %an independent and bounded  
a zero-mean random noise bounded, independent across rounds and arms.

\paragraph{Corrupted stochastic regime.}
The loss satisfies $\ell_t \spr*{x, a} \coloneqq \inp*{x,\theta_{t, a}} + \epsilon_{t, a}$, %\red{[If we mean that $\epsilon_t$ is a function of $ \spr*{X_t, a}$, then we should write it explicitly in the description that follows]} 
, where $\epsilon_{t, a}$ is again a zero-mean random noise bounded in $\sbr*{-1, 1}$. In this regime, the adversary may corrupt the parameters over time, but only within a limited budget: there exists fixed but unknown vectors $\spr*{\theta_a}_{a \in \sbr*{K}}$ and a constant $C > 0$ such that $\sumtT \max_{a \in \sbr*{K}} \norm*{\theta_{t, a} - \theta_a}_2 \leq C$. The extreme cases $C = 0$ and $C = T$ recover, respectively, the stochastic regime and the adversarial regime (up to the presence of random noise).

Let $\Pi$ denote the set of deterministic policies $\pi\colon \cX \mapsto \sbr*{K}$. We define the best policy in hindsight $\pi_T^\star$ by
%
\begin{equation*}
    \pi_T^\star\colon x \in \cX \mapsto \argmin_{a \in \sbr*{K}} \bbE \sbr*{\sumtT \ell_t \spr*{x, a}},
\end{equation*}
%
where potential randomness of the loss distribution. The learners' ojective is to minimize the expected cumulative regret against $\pi_T^\star$,
%
\begin{align} \label{eq::regret_def}
    R_T &= \bbE \sbr*{\sumtT \spr*{\ell_t \spr*{X_t, A_t} - \ell_t \spr*{X_t, \pi_T^\star \spr*{X_t}}}}\\
    &\quad+ \bbE \sbr*{\sumtT c \cdot \abs*{O_t}}, \nonumber
\end{align}
%
where the expectation here additionally includes the learner's internal randomization.

\DB{If we re-write the proof with ghost sample, explain here}

\paragraph{Additional definitions.} In the (corrupted) stochastic regime, we further define, for any context $x \in \cX$,
%
\begin{equation*}
    \deltamin \spr*{x} \coloneqq \min_{a \neq \pi_T^\star \spr*{x}} \inp*{x, \theta_a - \theta_{\pi_T^\star \spr*{x}}}
\end{equation*}
%
and the minimum sub-optimality gap
%
\begin{equation*}
    \deltamin \coloneqq \min_{x \in \cX} \deltamin \spr*{x}.
\end{equation*}
%
If the distribution $\cD$ over contexts is discrete, then $\deltamin$ is always strictly positive if all arms have distinct parameters. However, in the case where $\cD$ is continuous, it is possible that $\deltamin = 0$. In such cases, stochastic regret guarantees depending on $\deltamin^{-1}$ become vacuous. Nonetheless, the adversarial regret bounds remain valid regardless of the value of $\deltamin$.

We denote $\cH_t = \sigma \spr*{X_s, A_s, O_s, \scbr*{l_s \spr*{X_s, o}}_{o \in O_s}, s \leq t}$ the filtration generated by all past contexts, actions, and observed losses. Finally, we use equivalently the notation $a = \cO \spr*{b}$ or $a \lesssim b$ when there exists a constant $\omega > 0$ such that $a \leq \omega b$, where $\omega$ is independent of the following problem-dependent quantities: $T, d, K, \Sigma, \cD, C, \deltamin$.
\section{ALGORITHM} \label{sec::algorithm}

As is standard in the best-of-both-worlds literature, our algorithm builds on the \emph{Follow-the-Regularized-Leader} (FTRL) framework \citep[see, \eg,][Sec. 2.3]{ShalevShwartz12}. This general principle is characterized by three key design choices: a \emph{loss estimator}, a \emph{learning-rate schedule}, and an appropriate \emph{regularizer}.

To obtain loss estimates adapted to the linear contextual setting, we follow the approach of \citet{BOBWlinear}, constructing importance-weighted regression estimates of the losses. For computational efficiency, we employ the \emph{Matrix Geometric Resampling (MGR)} method \citep{neu2013efficient, bartok2014partial, BOBWlinear}, which guarantees tractability while controlling both the bias and variance of the estimates (see also \citealp{neu2016exploration}).

The other components of our algorithm are more directly inspired by Algorithm~2 of \citet{BOBWhardproblems}, which addresses the best-of-both-worlds problem for multi-armed bandits with paid observations. In particular, we adopt their use of a Tsallis entropy regularizer \antoine{why?}, an adaptive learning-rate schedule, and the computation of an \emph{observation probability} that is uniform across arms. This probability is derived from the sampling probability vector produced by FTRL. This idea to use distinct observation and sampling probabilities originates from the initial work of \citet{SeldinS14}.

In the following, we detail the components of our algorithm for linear contextual bandits with paid observations. The pseudo-code can be found in Algorithm~\ref{alg::FTRL_bobw}.

\paragraph{Sampling distribution (FTRL).} We recall that, at each round $t \geq 1$, the learner observes a context vector $X_t$, and must choose an action $A_t \in \sbr*{K}$. As a first step, our algorithm computes a sampling distribution $q_t \spr*{\cdot \given X_t} \in \Delta_K$, where $\Delta_K$ denotes the $\spr*{K-1}$-dimensional probability simplex. Following \citet{BOBWhardproblems}, given a context $x$, this distribution is obtained through the \emph{Follow-the-Regularized-Leader} (FTRL) principle, by solving the optimization problem \antoine{minus missing in front of the second entropy?}
%
\begin{equation} \label{eq::FTRL}
    q_t \spr*{\cdot \given x} \in \argmin_{q \in \Delta_K} \scbr*{\sum_{s=1}^{t-1} \inp*{q, \wt{\ell}_s \spr*{x}} + \psi_t \spr*{q} + \bar{\beta} H_{\bar{\alpha}} \spr*{q}}.
\end{equation}
%
Note that $x \mapsto q_t \spr*{\cdot \given x}$ is $\cH_{t-1}$-measurable. This formulation involves the following components:
%
\begin{itemize}
    \item \textbf{Loss estimates.} For each round $s \leq t-1$,
    %
    \begin{equation} \label{eq::estimator}
        \widetilde \ell_s \spr*{x} \coloneqq \spr*{\inp*{x, \wt{\theta}_{s, 1}}, \ldots, \inp*{x, \wt{\theta}_{s, K}}}\transpose,
    \end{equation}
    %
    where $\wt{\theta}_{s,a}$ is an estimator of the linear loss parameter $\theta_{s, a} \in \bbR^d$ (see Eq.~\eqref{eq::MGR_estimate}).

    \item \textbf{Regularizer.} We use the Tsallis entropy, with
    %
    \begin{equation*}
        \psi_t \spr*{q} \coloneqq - \frac{H_\alpha \spr*{q}}{\eta_t}, \text{ for } H_{\alpha} \spr*{q} \coloneqq \textcolor{red}{\frac{1}{\alpha - 1}} \sum_{a=1}^{K} \spr*{q_a^\alpha - q_a},
    \end{equation*}
    %
    \db{Should be $\frac{1}{\alpha - 1}$, I see the error is propagated from \cite{BOBWhardproblems} (below Eq. (10)), but I guess in their derivations they then use the right one.}
    where $\eta_t > 0$ is the learning rate at time $t$, and we fix $\alpha \coloneqq 1 - \spr*{\log K}^{-1}$. For convenience, we also define $\beta_t \coloneqq 1/\eta_t$.

    \item \textbf{Additional parameters.} We set $\bar{\alpha} \coloneqq 1 - \alpha$ and
    %
    \begin{equation*}
        \bar{\beta} \coloneqq \tfrac{32 K d \sqrt{c}}{\spr*{1 - \alpha}^2 \sqrt{\beta_1} \min \spr*{1, \lambda_{\min}}},
    \end{equation*}
    %
    where $c, K$, and $\lambda_{\text{min}}$ are as introduced in Section~\ref{sec::setting}. The term $\beta_1 = \eta_1^{-1}$ is introduced here in order to simplify some parts of the analysis, since we will define the learning rate such that $\beta_t \geq \beta_1$ holds for all time steps $t\geq 1$.
\end{itemize}
%
The definition of the FTRL distribution in Eq.~\eqref{eq::FTRL} follows Algorithm~2 of \citet{BOBWhardproblems}, with two key modifications. The first, as previously discussed, is the use of loss estimates specifically adapted to the linear contextual structure of our setting.

The second is the value of $\bar{\beta}$ before the second regularization term, which we use in the analysis to control the evolution of $H_\alpha \spr*{q_t}$ between rounds (see Lemma~\ref{lem::cond_bound_ht}), in particular at the beginning of the interaction (since this term does not scale up with $t$). This value is adjusted by the parameter $\lambda_{\min}$ to account for the impact of the context distribution in the analysis.

\paragraph{Estimation of the linear losses.}
We rely on a standard importance-weighted estimator, adapted from \citet{BOBWlinear}. The key modification is that, instead of using the sampled action, we use the actions that are \emph{observed} (if any) at round $t$. Specifically, for $t \geq 1$ and $a \in \sbr*{K}$, we could estimate $\theta_{t, a}$ by
%
\begin{equation} \label{eq::thetahat}
    \wh \theta_{t, a} \coloneqq \Sigma_{t, a}^{-1} X_t \ell_t \spr*{X_t, a} \ind_{\scbr*{a \in O_t}},
\end{equation}
%
where $\Sigma_{t, a} \coloneqq \bbE \sbr*{\ind_{{a \in O_t}} X_t X_t\transpose \given \textcolor{red}{\cH_{t-1}}}$ \antoine{maybe just define it independently of $a$?}. However, computing $\Sigma_{t, a}^{-1}$ exactly is computationally impractical for two reasons. First, matrix inversion at every round costs $\cO \spr*{d^3}$ operations, which becomes prohibitive in high dimensions. Second, evaluating $\Sigma_{t, a}$ itself may be extremely costly: even in the discrete-context case, it requires computing observation probabilities for all possible contexts, with complexity at least $\cO \spr*{\abs*{\cX}}$, and moreover presupposes full knowledge of the context distribution.

To circumvent this issue, we approximate $\Sigma_{t, a}^{-1}$ using the \emph{Matrix Geometric Resampling} (MGR) procedure, described in Algorithm~\ref{alg:MGR} (Appendix). Computationally, MGR only requires sampling $M_t$ contexts independently from $\cD$, evaluating their observation probabilities (\ie, those the algorithm would assign if the context were observed at round $t$), and performing basic algebraic operations. This reduces the dependence of the cost from $\abs*{\cX}$ to $\cO \spr*{\log \spr*{T}}$, while only requesting access to a sampler of $\cD$.
%to $\cO(d^{2}\log(T))$

Accordingly, the estimator used in our algorithm is
%
\begin{equation}\label{eq::MGR_estimate}
    \wt{\theta}_{t, a} \coloneqq \Sigma_{t, a}^+ X_t \, \ell_t \spr*{X_t, a} \ind_{\scbr*{a \in O_t}},    
\end{equation}
%
where $\Sigma_{t, a}^+$ is the approximation of $\Sigma_{t, a}^{-1}$ returned by the MGR routine. \textcolor{red}{Denote $\ptmin = \min p_t$.} Guided by our analysis, we set the number of MGR iterations to
%
\begin{equation} \label{eq::Mt}
    M_t \coloneqq \left\lceil \frac{4 K}{\textcolor{red}{\ptmin} \lambda_{\min}} \ln \spr*{t} \right\rceil,
\end{equation}
%
\antoine{Can probably replace $\ptmin$ by $\gamma_t$ with uniform mixing.}
which ensures sufficiently accurate approximation of $\Sigma_{t, a}^+$. Compared to \citet{BOBWlinear}, where the bias of the estimator is controlled via a forced exploration rate, in our setting this role is played by the observation probability $p_t$.

\paragraph{Observation probability.} Since observing each arm incurs a fixed cost $c$, the observation probability $p_t$ must balance variance reduction with cost. \textcolor{red}{For any context $x$}, we define
%
\begin{align} \label{eq::def_zt_ut}
    z_t \spr*{x} &\coloneqq \frac{4 c K d^2}{\spr*{1 - \alpha} \lambda_{\min}^2} \spr*{q_{t*} \spr*{x}^{2 - \alpha} + \sum_{i \neq I_t} q_t \spr*{i \given x}^{2 - \alpha}}, \nonumber\\
    u_t \spr*{x} &\coloneqq \frac{8 d \max \spr*{c, 1}}{\spr*{1 - \alpha} \lambda_{\min}} q_{t*} \spr*{x}^{1 - \alpha}, \text{ where}\\
    I_t \spr*{x} &\coloneqq \argmax_{i \in \sbr*{K}} q_{t} \spr*{i \given x}, \text{ and } \nonumber\\
    q_{t*} \spr*{x} &\coloneqq \min \scbr*{q_t \spr*{I_t \spr*{x} \given x}, 1 - q_t \spr*{I_t \spr*{x} \given x}}. \nonumber
\end{align}
\DB{$d$ here introduced artificially, redo later.}

Compared to Algorithm~2 in \citet{BOBWhardproblems}, we have modified the definitions of the quantities $z_t$ and $u_t$ to include the $\lambda_{\min}$ and $d$ terms, which becomes necessary to appropriately control the variance of importance-weighted losses. For a learning rate $\eta_t$, we then define the observation probability as
%
\begin{equation} \label{Rule1}
    p_t \spr*{x} \coloneqq \min \scbr*{\frac{\sqrt{z_t \spr*{x} \eta_t} + u_t \spr*{x} \eta_t}{c K}, 1}.
\end{equation}
%
This tuning seems to differ from the one proposed in Eq.~93 of \citet{BOBWhardproblems} for their BoBW algorithm in the MAB with paid observations setting. As we explain in Section~\ref{sec::regret}, our choice avoids a factor $\spr{\tfrac{1}{cK} + cK}$ in the regret bound, which would otherwise render the guarantee vacuous when $c$ is very small. Moreover, Eq.~\eqref{eq::def_zt_ut} shows that without this inverse scaling in $c$, the observation probability would converge to zero for small $c$ under a fixed sampling probability, which is an unintuitive and undesirable behavior.

The fact that the probability $p_t$ is uniform across arms has two important consequences for the MGR scheme. First, it removes the need for the forced exploration mechanism used in \citet{BOBWlinear} to control the bias (see their Lemma~9), and instead leads to a different result, formalized in our Lemma~\ref{lem::MGRbound}. Second, since $\Sigma_{t, a}$ is identical for all arms, we only need to compute a single pseudo-inverse $\Sigma_t^+$ per round. As a result, MGR only needs to be executed once at each time step, significantly reducing the overall computational cost.


\paragraph{Learning rate.} The learning rate $\eta_t$ balances stability and adaptivity of FTRL, and is chosen to ensure optimal regret in both regimes. We follow Rule 2 of the framework presented in \citet{BOBWhardproblems} and use the update rule
%
\begin{equation} \label{Rule2}
    \frac{1}{\eta_{t+1}} = \frac{1}{\eta_t} + \frac{1}{h_t \spr*{X_t}} \spr*{2 \sqrt{z_t \spr*{X_t} \eta_t} + u_t \spr*{X_t} \eta_t},
\end{equation}
%
where $h_t \spr*{X_t}$ denotes the entropy $H \spr*{q_t \spr*{\cdot \given X_t}}$. For notational convenience we set $\gamma_t \spr*{x} = c K p_t \spr*{x}$ \antoine{where is it used?}. We also choose $\eta_1$ to ensure that $p_t \leq \frac12$ for all time steps,
%
\begin{equation} \label{eq::eta1}
    \eta_1 = \frac{\spr*{1 - \alpha} \lambda_{\min}^2}{64 \max \spr*{c, 1} K}.
\end{equation}

\begin{algorithm}
	\caption{FTRL for linear contextual bandits with paid observations}
    \label{alg::FTRL_bobw}
	\begin{algorithmic}[1]
        \STATE {\bfseries Input:} $K$ arms, cost $c$, minimum eigenvalue $\lambda_{\min}$.
        \STATE Initialize $\eta_1$ as in Eq.~\eqref{eq::eta1}, and for any arm $a \in \sbr*{K}$, set $\wt{\theta}_{0, a} = 0$.
		\FOR{$t = 1, 2, \ldots, T$}
			\STATE Observe $X_t$ and compute $q_t \spr*{\cdot \given X_t}$ as in Eq.~\eqref{eq::FTRL}.
            \STATE Sample $A_t \sim q_t \spr*{\cdot \given X_t}$.
            \STATE Compute $p_t \spr*{X_t}$ as in Eq.~\eqref{Rule1}.
            \STATE For any $a$, observe $\ell_t \spr*{X_t, a}$ with prob.\! $p_t \spr*{X_t}$.
            \STATE Suffer the loss $\ell_t \spr*{X_t, A_t} + c \abs*{O_t}$.
            \STATE Update $\eta_t$ to $\eta_{t+1}$ according to Eq.~\eqref{Rule2}.
            \STATE For any $a$, compute and store $\wt \theta_{t, a}$ via Alg.~\ref{alg:MGR}.
            \STATE Compute and store $\Sigma_t^+$ via MGR (see Alg.~\ref{alg:MGR}) with $M_t$ iterations.
		\ENDFOR
	\end{algorithmic} 
\end{algorithm}

\paragraph{Computation time and memory.} The total space and time complexity of Algorithm~\ref{alg::FTRL_bobw} are respectively $\cO \spr*{T d^2}$ and $\cO \spr*{K^2 T^2 d^2 \log T}$. Details can be found in Appendix~\ref{AppendixAlgAnalysis}.

\section{REGRET ANALYSIS} \label{sec::regret}

We now introduce the main theoretical result of this work, which is that Algorithm~\ref{alg::FTRL_bobw} achieves Best-of-Both-Worlds regret guarantees in the setting of linear bandits with paid observations, under the assumptions introduced in Section~\ref{sec::setting}.

\db{We have to change the dimension dependency after the fix.}
\begin{restatable}{theorem}{MainTheorem}\label{thm::main}
    In the adversarial regime, the regret of Algorithm~\ref{alg::FTRL_bobw} satisfies
    \begin{align*}
        R_T &\lesssim \spr*{\frac{c K d^2 \log K}{\lambda_{\min}^2}}^{1/3} T^{2/3} \\
        &\quad+ \sqrt{\frac{\max \spr*{c, 1} d \log K \cdot T}{\lambda_{\min}}} + \kappa
    \end{align*}
    %
    with \antoine{$\log KT$ or $T \log K$?}
    %
    \begin{align*}
        \kappa &= \sqrt{\frac{c K d^2 \log K}{\lambda_{\min}^2}} + \frac{\max \spr*{c, 1} d \log K}{\lambda_{\min}}\\
        &\quad+ \frac{\max \spr*{c, 1} K \log K}{\lambda_{\min}^2} + \frac{32 K d \sqrt{c}}{\spr*{1 - \alpha}^2 \sqrt{\beta_1} \min \spr*{1, \lambda_{\min}}}.
    \end{align*}
    %
    while in the corrupted stochastic regime with corruption level $C$, it satisfies
    %
    \begin{align*}
        R_T &\lesssim \frac{d \sqrt{\max \spr*{c, 1} K \log K}}{\lambda_{\min} \deltamin^2} \cdot \log \spr*{T \deltamin^3} \\
        &\quad+ \spr*{\frac{C^2 d \sqrt{\max \spr*{c, 1} K \log K}}{\lambda_{\min}\deltamin^2} \cdot \log \spr*{\frac{T \deltamin}{C}}}^{1/3}\\
        &\quad+ \kappa + \kappa', \; \text{where we further define} \\
        \kappa' &= \spr*{\spr*{\frac{c K d^2 \log K}{\lambda_{\min}^2}}^{1/3} +  \sqrt{\frac{\max \spr*{c, 1} d \log K}{\lambda_{\min}}}}\\
        &\quad\times \spr*{\frac{1}{\deltamin^3} + \tfrac{C}{\deltamin}}^{2/3}.
    \end{align*}
\end{restatable}

This result shows that Algorithm~\ref{alg::FTRL_bobw} achieves the minimax-optimal $\cO \spr*{T^{2/3}}$ regret in the adversarial regime, while smoothly adapting to the (possibly corrupted) stochastic regime with logarithmic dependence on $T$ when $C = 0$. These bounds match the known lower bounds from \citet{paidobservations}, which applies to our setting since it encompasses the standard multi-armed bandit (by taking $d=1$ and $X_t=1$ a.s.), and extend the Best-of-Both-Worlds (BoBW) framework of \citet{BOBWhardproblems} to the setting of linear bandits.

While the dependence in $T$ is thus known to be optimal, the optimal dependence in other problem-specific parameters remains unknown, as this is the first work to address this setting. However, since our algorithm builds upon and generalizes both Algorithm~2 from \citet{BOBWlinear} and Algorithm~2 from \citet{BOBWhardproblems}, we can compare our regret bounds to theirs, even if the settings do not perfectly align. We consider first the limiting case where $c \to 0$, corresponding to the full-information setting, in which all losses are observed. In this regime, the first term of the adversarial regret bound vanishes, and we have %the dominant term becomes the one scaling as $\sqrt{dT}$, and so
%
\begin{equation*}
    R_T \lesssim \sqrt{\frac{d T \log \spr*{K}}{\lambda_{\min}}}.
\end{equation*}
%
This matches, up to logarithmic factors, the adversarial regret bound established for Algorithm~2 in \citet{BOBWlinear}, namely
%
\begin{equation*}
    R_T\lesssim \sqrt{T \spr*{d + \frac{\log T}{\lambda_{\min}}} K \log K \log T}.
\end{equation*}
%
In our case, the factor $K$ is replaced by $\log K$, which reflects the full-information nature of our setting, a standard improvement in such regimes. However, in the stochastic regime, our regret exhibits an additional $\frac{1}{\deltamin}$ factor compared to the full-information bounds in \citet{BOBWlinear}. But on the countrary, our algorithm has a better $\log T$ dependence, thus our bound is better if $T$ is significantly larger than $\frac{1}{\deltamin}$. However, we do not know whether our improved $\log T$ dependency stems from being in the full-information setting or from other factors. We can at least observe that the dependence on the setting-specific parameters $d$ and $\lambda_{\min}$ in our bounds matches that of their Algorithm~2.

Another useful comparison is to consider the special case $d = 1, \cX = \scbr*{1}$, %swhere the context space has size one, which implies \( \lambda_{\min} = 1 \).
in which case we recover the setting of \citet{SeldinS14}.
%, similar to Algorithm~2 from \citep{BOBWhardproblems}.
From their Corollary~17,  Algorithm~2 of \citet{BOBWhardproblems} obtain an adversarial regret bound of
%
\begin{equation*}
    R_T \lesssim \spr*{\spr*{c K}^{1/3} T^{2/3} \spr*{\log K}^{1/3}},
\end{equation*}
%
which is exactly the scaling that we obtain with Theorem~\ref{thm::main} in this setting. This observation furthermore still holds in the stochastic setting.

These comparisons suggest that, while we can not establish optimality in general due to the lack of known lower bounds, our algorithm can be viewed as a strict generalization of the approach in \citet{BOBWhardproblems} for bandits with paid observations, since we recover their guarantees in this setting. Moreover, since the dependencies in $d$ and $\lambda_{\min}$ are known to be optimal compared to previous approaches when $c = 0$, this further supports the relevance of our design beyond prior approaches.

A detailed proof of the theorem can be found in Appendix~\ref{AppendixRegret}. In the following, we present the main steps of the proofs, highlighting the technical arguments that required to be adapted from the existing frameworks.

\begin{proof}[Proof sketch]
    As a preliminary step of the analysis, we isolate the difficulty induced by the use of (biased) MGR estimates (Eq.~\eqref{eq::MGR_estimate}) instead of using the unbiased estimators from Eq.~\eqref{eq::estimator}. Following the proof technique of \citet{BOBWlinear}, we introduce an auxiliary game where these estimators are treated as unbiased, and for which the regret would thus become
    %
    \begin{equation*}
        \tilde{R}_T \coloneqq \bbE \sbr*{\sumtT \inp*{X_t, \tilde{\theta}_{t, A_t}} - \inp*{X_t, \tilde{\theta}_{t, \pi^\star \spr*{X_t}}}}.
    \end{equation*}
    %
    We can verify that the actual regret of our algorithm thus satisfies
    %
    \begin{equation*}
        R_T \leq \tilde{R}_T + 2 \sumtT \max_{a \in \sbr*{K}} \abs*{\bbE \sbr*{\inp*{X_t, \tilde{\theta}_{t, a} - \theta_{t, a}}}}.
    \end{equation*}
    %
    Then, in Lemma~\ref{lem::boundBias} we prove that the second term of this upper bound can be upper bounded by a constant, independent of all problem parameters. In the following, we thus focus on upper bounding $\tilde{R}_T$. We write the following proof steps with the notation $R_T$, with an abuse of notation, since previous result showed that both terms have the same scaling in $T$.

    The remainder of the analysis builds on the general framework introduced by \citet{BOBWhardproblems} to build Best-of-Both-Worlds algorithms for problems with minimax regret scaling with $T^{2 / 3}$, and in particular their instantiation of this framework to tackle standard multi-armed bandit with paid observations (without the linear contextual structure). Our first contribution is an adaptation of their Theorem 7 to accommodate the linear contextual structure, that we introduce below.
    %
    \begin{restatable}[Adaptation of Theorem~7 of \citealp{BOBWhardproblems}]{lemma}{adath} \label{lem::adapt_tsuchaya}
        Suppose that Algorithm~\ref{alg::FTRL_bobw} satisfies the following conditions in the adversarial regime:
        %
        \begin{enumerate}[label=(\roman*),ref=(\roman*)]
            \item $R_T \leq \sumtT \bbE \sbr*{\spr*{\tfrac{1}{\eta_t} - \tfrac{1}{\eta_{t - 1}}} h_t + \tfrac{z_t \eta_t}{\gamma_t} + \gamma_t} + \bar{\beta} \bar{h}$,
            
            \item $\bbE \sbr*{h_{t+1} \given \cH_t} \leq 2 \bbE \sbr*{h_t \mid \cH_{t - 1}}$ for all $t \geq 1$.
        \end{enumerate}
        %
        Then the regret can be bounded as
        %
        \begin{equation*}
            R_T \lesssim \spr*{z_{\max} h_1}^{1/3} T^{2/3} + \sqrt{u_{\max} h_1 T} + \kappa,
        \end{equation*}
        %
        where
        %
        \begin{align*}
            z_{\max} &= \max_{t \in \sbr*{T}} z_t \leq 4 c K \log K \frac{1}{\lambda_{\min}^2},\\
            u_{\max} &= \max_{t \in \sbr*{T}} u_t \leq 4 \max \spr*{c, 1} \log K \frac{1}{\lambda_{\min}},
        \end{align*}
        %
        and
        %
        \begin{equation*}
            \kappa \coloneqq \sqrt{z_{\max} \eta_1} + u_{\max} \eta_1 + \tfrac{h_1}{\eta_1} + \bar{\beta} h_{\max}.
        \end{equation*}
        %
        Moreover, if Algorithm~\ref{alg::FTRL_bobw} satisfies the following conditions in the stochastic regime: there exists a constant $\rho > 0$ such that, for any $t \geq 1$,
        %
        \begin{enumerate}[label=(\roman*),ref=(\roman*),start=3]
            \item $\sqrt{z_t h_t} \leq \sqrt{\rho}\spr*{1 - q_t \spr*{\pi_T^\star \spr*{X_t} \given X_t}}$, and
            \item $u_t h_t \leq \rho\,\spr*{1 - q_t \spr*{\pi_T^\star \spr*{X_t} \given X_t}}$, 
        \end{enumerate}
        %
        then, for $T \geq \tau \coloneqq \tfrac{1}{\deltamin^3} + \tfrac{C}{\deltamin}$ it holds that
        %
        \begin{align*}
            R_T &\lesssim \frac{\rho}{\deltamin^2} \log \spr*{T \deltamin^3} + \spr*{\tfrac{C^2 \rho}{\deltamin^2} \log \spr*{\tfrac{T \deltamin}{C}}}^{1/3}
            + \kappa'
        \end{align*}
        %
        with
        %
        \begin{equation*}
            \kappa' \coloneqq \kappa + \spr*{\spr*{z_{\max} h_1}^{1/3} + \sqrt{u_{\max} h_1}} \spr*{\tfrac{1}{\deltamin^3} + \tfrac{C}{\deltamin}}^{2/3}.
        \end{equation*}
    \end{restatable}
    %
    While Lemma~\ref{lem::adapt_tsuchaya} adapts Theorem 7 from \citet{BOBWhardproblems}, it differs in several significant aspects. First, condition~(i) is new and replaces conditions~(i)--(ii) in the original theorem, and both lead to a similar proof structure, our condition better adjust the framework to our setting. Second, condition~(ii) is a relaxed reformulation of condition~(iii) in \citet{BOBWhardproblems}, which is necessary to handle the stochasticity of contexts in our setting. With careful use of the tower rule, we show that this weaker assumption is sufficient for the regret analysis. Finally, conditions~(iii) and (iv) are reformulations of conditions~(iv) and (v) from \citet{BOBWhardproblems}, and the corresponding proof techniques carry over with only little modifications. The detailed proof of this lemma is deferred to Appendix~\ref{Appendix}.

    To establish Theorem~\ref{thm::main}, it then suffices to verify that Algorithm~\ref{alg::FTRL_bobw} satisfies each of the four conditions. 

    Condition~(i) follows from the standard FTRL regret decomposition: the stability term bound is direct to obtain, while the penalty term is controlled using Lemma~\ref{lem::ub_ftrl_technical} (in Appendix)% on $|l_t\eta_t|$
    , which is similarly to the proof of \citet[Theorem 8]{BOBWhardproblems}. 

    We prove condition~(ii) in Lemma~\ref{lem::cond_bound_ht}. The proof consists in applying Lemma 15 from \citet{BOBWhardproblems} (restated as Lemma~\ref{lem::ftrl_smooth_bound}) for each fixed context, and to conclude via linearity of expectation. A key challenge arises from the fact that, in our setting, we have the bound $\bbE\left[\langle X_t, \hat{\theta}_{t, a} \rangle^2\right] \leq \frac{1}{\lambda_{\min}^2 p_t}$, which contrasts with the original bound $\bbE\left[\ell_t^2\right] \leq \frac{1}{p_t}$ in the non-contextual case. Since Lemma~\ref{lem::ftrl_smooth_bound} only accommodates a constant upper bound, this discrepancy required a careful adjustment of several parameters, specifically $u_t$ and $\bar{\beta}$, which represents a slight modification in the precise behavior of the algorithm.

    Finally, conditions~(iii) and (iv) are verified by combining entropy bounds from \citet{BOBWhardproblems} with direct control of the variance-like quantities $z_t$ and $u_t$, thereby linking them to the optimal action probability.  

    Together, these arguments ensure that Algorithm~\ref{alg::FTRL_bobw} satisfies the assumptions of Lemma~\ref{lem::adapt_tsuchaya}, which directly yields the regret guarantees stated in Theorem~\ref{thm::main}.

    The full derivations and supporting lemmas are deferred to Appendix~\ref{AppendixRegret}, where we carefully establish that each condition of the lemma holds in our setting. 
\end{proof}

While the definition of $p_t$ in \citet{BOBWhardproblems} differs from ours by a factor $\spr*{c K}^{-1}$, this appears to be a simple typo in their presentation. Indeed, their analysis assumes $p_t = \frac{1}{c K} \spr*{\sqrt{z_t \eta_t} + u_t \eta_t}$, even though the statement of their Algorithm~2 defines $p_t \coloneqq \sqrt{z_t \eta_t} + u_t \eta_t$. %However, this normalization is necessary to ensure consistency with their analysis, which assumes \( \gamma_t = cK p_t = \sqrt{z_t \eta_t} + u_t \eta_t \). 
We can use this observation to comment on the optimality of the tuning of $p_t$ with respect to the analysis used to derive BoBW regret bounds for our algorithm.

Indeed, a step in the analysis (see Eq.~\eqref{eq::critical step}) involves the quantity $\gamma_t' \coloneqq \gamma_t - \frac{u_t}{\beta_t}$. With our definition, this yields $\gamma_t' = \sqrt{z_t / \beta_t}$, while using the unnormalized $p_t$ (without $1 / \spr*{cK}$) gives
%
\begin{equation*}
    \gamma_t' = c K \sqrt{z_t \eta_t} + \spr*{c K - 1} u_t \eta_t \geq c K \sqrt{z_t \eta_t},
\end{equation*}
%
assuming $c K \geq 1$. This leads to the bound
%
\begin{align*}
    \sumtT \bbE \sbr*{\frac{z_t \eta_t}{\gamma_t'} + \gamma_t} &\leq \sumtT \bbE \sbr*{\frac{1}{c K}\sqrt{\frac{z_t}{\beta_t}} + c K \spr*{\sqrt{\frac{z_t}{\beta_t}} + \frac{u_t}{\beta_t}}} \\
    &\leq \spr*{\frac{1}{c K} + c K} \sumtT \bbE \sbr*{2 \sqrt{\frac{z_t}{\beta_t}} + \frac{u_t}{\beta_t}}.
\end{align*}
%
The factor $\spr*{c K}^{-1} + c K$ then propagates through the analysis and degrades the regret bound. More generally, an overestimation of $p_t$ by a multiplicative factor $\omega$ leads to a regret that is worsened by a factor proportional to $\omega + \omega^{-1}$, so $\omega = 1$ (our tuning) is optimal.
\section{DISCUSSION} \label{sec::discuss}

We proposed an algorithm achieving BoBW regret guarantees in the setting of \emph{linear contextual bandits with paid observations}, with explicit scaling in problem dimensions ($d$, $K$) and parameters ($\lambda_{\text{min}}$, $\deltamin$, $c$).

However, an important limitation, shared with the analysis of Algorithm~2 from \citet{BOBWlinear}, arises in the stochastic setting when the context space is continuous. In such cases, the quantity $\deltamin$ is often zero, which implies that the regret bound remains at $\Theta \spr*{T^{2/3}}$, even though the environment is stochastic and should, in principle, allow for better rates. This issue also affects discrete but finely spaced context spaces, where $\deltamin > 0$ but can be arbitrarily small, leading to overly pessimistic bounds in practice. Nevertheless, \citet{BastaniBK21} demonstrates that under suitable regularity conditions on the context distribution, it is possible to achieve logarithmic regret in continuous settings without any dependence on $\deltamin$. Extending such ideas to our setting, and combining them with BoBW-style guarantees, could lead to improved regret bounds, potentially polylogarithmic or polynomially better than $\sqrt{T}$ or $T^{2/3}$. We believe this is a promising direction for future work.

Finally, as previously discussed, since this setting is novel, there are currently no lower bounds specifically tailored to it. Existing lower bounds only apply to simplified or special cases of our setting. Developing minimax and stochastic lower bounds that are adapted to this setting, precisely capturing all dimensions and parameters, would therefore be an interesting contribution to improve the understanding of this setting.

% Acknowledgements should only appear in the accepted version.
% \section*{Acknowledgements}

% \textbf{Do not} include acknowledgements in the initial version of the paper
% submitted for blind review.

% If a paper is accepted, the final camera-ready version can (and usually should)
% include acknowledgements.  Such acknowledgements should be placed at the end of
% the section, in an unnumbered section that does not count towards the paper
% page limit. Typically, this will include thanks to reviewers who gave useful
% comments, to colleagues who contributed to the ideas, and to funding agencies
% and corporate sponsors that provided financial support.

\section*{Impact Statement}

This paper presents work whose goal is to advance the field of Machine
Learning. There are many potential societal consequences of our work, none
which we feel must be specifically highlighted here.

% In the unusual situation where you want a paper to appear in the
% references without citing it in the main text, use \nocite


\bibliography{biblio}
\bibliographystyle{icml2026}

%%%%%%%%%%%%%%%%%%%%%%%%%%%%%%%%%%%%%%%%%%%%%%%%%%%%%%%%%%%%%%%%%%%%%%%%%%%%%%%
%%%%%%%%%%%%%%%%%%%%%%%%%%%%%%%%%%%%%%%%%%%%%%%%%%%%%%%%%%%%%%%%%%%%%%%%%%%%%%%
% APPENDIX
%%%%%%%%%%%%%%%%%%%%%%%%%%%%%%%%%%%%%%%%%%%%%%%%%%%%%%%%%%%%%%%%%%%%%%%%%%%%%%%
%%%%%%%%%%%%%%%%%%%%%%%%%%%%%%%%%%%%%%%%%%%%%%%%%%%%%%%%%%%%%%%%%%%%%%%%%%%%%%%
\newpage
\appendix
\onecolumn


\section{Comments Dorian}
\DB{Updated!}

Comments after visit in Oxford, Feb 26:
\begin{itemize}
    \item We managed to move forward with the ghost sample trick.
    \item To get the desired bound on the stability term we had to redefine all quantities in an average (over context) sense. 
    \item It seems that we could at least get the $T^{2/3}$ bound this way. Not clear about the stochastic bound
	\item Having context-dependent observation probabilities seem quite hard. We would need to finely examine the smoothness of this prob. This might be an interesting direction, but what assumptions are needed for this? 
\end{itemize}

To-DO:
\begin{itemize}
	\item Check the Bobw properties with the averaged parametrization
	\item If we can only get the $T^{2/3}$, benchmark with what a simpler adaptation of Exp3 + simpler time-dependent learning rate/exploration prob would get. 	
\end{itemize}

Questions:
\begin{itemize}
	\item It is not clear that we can relax the assumption that $\Delta_{\min}>0$ (even to "just" get the $T^{2/3}$). Doing so even in Yuko's setting might be interesting. It would be interesting to further refine what it can mean to be BOBW in a continuous contexts setting, maybe it might be cool to check the papers on Greedy to see if there are nice insights.
	\item Related: taking Kuroki's setting and use the stability/penalty matching framework of Taira with Tsallis entropy might be nice, and maybe we can get improved results. 
	\item Not clear what the lower bound should be (even in the full-info/no cost case), might be interesting too. 
\end{itemize}

\begin{itemize}
    \item Kuroki et al. have a worst stochastic bound in $(\log(T))^3+d(\log(T))^2$. Seems to be because how the exploration parameter should both be large enough for the MGR bias and at the same time depend tightly on the learning rate.
\end{itemize}

\section{Temp Antoine: Regret analysis}

    \subsection{Notation}

\antoine{Writing new notation here, propagate back into the main text later} For $\alpha \in \spr*{0, 1}$, we define the negative Tsallis entropy with parameter $\alpha$ over the nonnegative orthant $\bbRnnK$ as
%
\begin{equation*}
    H_\alpha \spr*{q} = \frac{1}{1 - \alpha} \sumaK \spr*{q_a - q_a^\alpha}.
\end{equation*}
%
\db{Mettre $1/\alpha$}
Note that $\nabla H_\alpha \spr*{q} = \frac{1}{1 - \alpha} \spr*{\bfone - \alpha q^{\alpha - 1}}$, and $\nabla^2 H_\alpha \spr*{x} = \alpha \Diag \spr*{q^{\alpha - 2}}$. At time $t$, for any context $x$ and arm $a$, the losses are defined as $\ell_t \spr*{x, a} = \inp*{x, \theta_{t, a}}$. We denote $p_t \spr*{x}$ the probability of observing each arm at time $t$ for context $x$, \ie, $p_t \spr*{x} = \bbP_t \sbr*{a \in O_t \given X_t = x}$, and we let $\Sigma_t = \bbE_t \sbr*{p_t \spr*{X_t} X_t X_t\transpose}$ the (conditional) covariance matrix of the contexts for the observed actions. We denote $\ptmin = \min_{x \in \cX} p_t \spr*{x}$. We consider the corresponding importance-weighted estimator $\wh{\theta}_{t, a} = \Sigma_t^{-1} X_t \ell_t \spr*{X_t, a} \, \ind_{\scbr*{a \in O_t}}$ and note that this is an unbiased estimator
%
\begin{align*}
    \bbE_t \sbr*{\wh{\theta}_{t, a}} &= \Sigma_t^{-1} \bbE_t \sbr*{X_t X_t\transpose \theta_{t, a} \ind_{\scbr*{a \in O_t}}}\\
    &= \Sigma_t^{-1} \bbE_t \sbr*{X_t X_t\transpose \theta_{t, a} \bbE_t \sbr*{\ind_{\scbr*{a \in O_t}} \given X_t}}\\
    &= \Sigma_t^{-1} \bbE_t \sbr*{p_t \spr*{X_t} X_t X_t\transpose} \theta_{t, a}\\
    &= \theta_{t, a},
\end{align*}
%
where we used the definition of the losses $\ell_t$, the law of total expectation, and the definitions of $p_t$ and $\Sigma_t$. We further denote $\wt{\theta}_{t, a}$ the MGR approximation of $\wh{\theta}_{t, a}$, defined as
%
\begin{equation*}
    \wt{\theta}_{t, a} = \whsigmaplust X_t \ell_t \spr*{X_t, a} \, \ind_{\scbr*{a \in O_t}},
\end{equation*}
%
where $\whsigmaplust$ is the random matrix output by the MGR procedure in Algorithm~\ref{alg:MGR}. This is a biased estimator, and assuming the procedure sampled $M_t$ additional contexts, we have \antoine{check the last equation carefully}
%
\begin{align*}
    \bbE_t \sbr*{\wt{\theta}_{t, a}} &= \bbE_t \sbr*{\whsigmaplust} \bbE_t \sbr*{X_t \ell_t \spr*{X_t, a} \, \ind_{\scbr*{a \in O_t}}}\\
    &= \bbE_t \sbr*{\whsigmaplust} \Sigma_t \cdot \theta_{t, a}\\
    &= \theta_{t, a} - \spr*{I - \frac12 \Sigma_t}^{M_t} \theta_{t, a}.
\end{align*}
%
We denote $\wh{\ell}_t$ and $\wt{\ell}_t$ the corresponding losses. For any $x$, any $t$, we consider the policies $q_t$ defined as
%
\begin{equation*}
    q_t \spr*{\cdot \given x} = \argmin_{q \in \Delta_K} \scbr*{\sum_{s=1}^{t-1} \inp*{q, \wt \ell_s \spr*{x}} + \psi_t \spr*{q} + \bar\psi \spr*{q}},
\end{equation*}
%
where we defined $\psi_t \spr*{q} = \frac{1}{\eta_t} H_\alpha \spr*{q}$ for some $\eta_t > 0$ to be defined, $\bar\psi \spr*{q} = \bar\beta H_{\bar\alpha} \spr*{q}$ with $\bar\alpha = 1 - \alpha$ and some $\bar\beta > 0$ to be defined, and $\wt \ell_s \spr*{x} \in \bbR^K$ is the vector of estimated losses at time $s$ for context $x$, with components $\wt \ell_s \spr*{x, a} = \inp*{x, \wt \theta_{s, a}}$ for any $a \in \sbr*{K}$.



    \subsection{Analysis}

Recall the definition of the regret
%
\begin{align*}
    R_T &= \bbE \sbr*{\sumtT \spr*{\ell_t \spr*{X_t, A_t} - \ell_t \spr*{X_t, \pi_T^\star \spr*{X_t}}}} + c\,\bbE \sbr*{\sumtT \abs*{O_t}}\\
    &= \bbE \sbr*{\sumtT \sumaK \spr*{q_t \spr*{a \given X_t} - \pi_T^\star \spr*{a \given X_t}} \inp*{X_t, \theta_{t, a}}} + c K\,\bbE \sbr*{\sumtT p_t \spr*{X_t}},
\end{align*}
%
where we used the fact that $\bbE_t \sbr*{\ell_t \spr*{X_t, a}} = \inp*{X_t, \theta_{t, a}}$ for any $a \in \sbr*{K}$ for the first term, and that $\bbE_t \sbr*{\abs*{O_t} \given X_t} = K p_t \spr*{X_t}$ for the second term, which holds because at time $t$, each arm $a$ is observed with probability $p_t \spr*{X_t}$.

We introduce a ghost sample $X_0 \sim \cD$ independent from $\cH_T$. Conditional on $\cH_{t-1}$, both $X_t$ and $X_0$ are \iid from $\cD$, and $p_t$ is $\cH_{t-1}$-measurable, hence, we have
%
\begin{equation*}
    \bbE_t \sbr*{p_t \spr*{X_t}} = \bbE_t \sbr*{p_t \spr*{X_0}}.
\end{equation*}
%
By Lemma~\ref{lem:ghost-sample-neu} and the fact that $\wh{\theta}_{t, a}$ is an unbiased estimator of $\theta_{t, a}$, \ie $\bbE_t \sbr{\wh{\theta}_{t, a}} = \theta_{t, a}$, we can further rewrite the regret as \antoine{there is a bug with the upper bound on the bias term at the end}
%
\begin{align*}
    R_T &= \bbE \sbr*{\sumtT \sumaK \spr*{q_t \spr*{a \given X_0} - \pi_T^\star \spr*{a \given X_0}} \inp*{X_0, \wh{\theta}_{t, a}}} + c K\,\bbE \sbr*{\sumtT p_t \spr*{X_0}}\\
    &= \bbE \sbr*{\sumtT \sumaK \spr*{q_t \spr*{a \given X_0} - \pi_T^\star \spr*{a \given X_0}} \inp*{X_0, \wt{\theta}_{t, a}}} + c K\,\bbE \sbr*{\sumtT p_t \spr*{X_0}}\\
    &\quad+ \bbE \sbr*{\sumtT \sumaK \spr*{q_t \spr*{a \given X_0} - \pi_T^\star \spr*{a \given X_0}} \inp*{X_0, \wh{\theta}_{t, a} - \wt{\theta}_{t, a}}}\\
    &\leq \bbE \sbr*{\sumtT \sumaK \spr*{q_t \spr*{a \given X_0} - \pi_T^\star \spr*{a \given X_0}} \inp*{X_0, \wt{\theta}_{t, a}}} + c K\,\bbE \sbr*{\sumtT p_t \spr*{X_0}}\\
    &\quad+ 2 \sumtT \max_{a \in \sbr*{K}} \abs*{\bbE \sbr*{\inp*{X_0, \wh{\theta}_{t, a} - \wt{\theta}_{t, a}}}}.
\end{align*}
%
For any context $x \in \cX$, we define the auxiliary regret
%
\begin{equation*}
    \wt R_T \spr*{x} \coloneqq \sumtT \sumaK \spr*{q_t \spr*{a \given x} - \pi_T^\star \spr*{a \given x}} \inp*{x, \wt{\theta}_{t, a}} + c K \sumtT p_t \spr*{x}
\end{equation*}
%
and the bias induced by MGR
%
\begin{equation*}
    \biasmgr \coloneqq \sumtT \max_{a \in \sbr*{K}} \abs*{\bbE \sbr*{\inp*{X_0, \wh{\theta}_{t, a} - \wt{\theta}_{t, a}}}}.
\end{equation*}
%
With this notation, we can write the inequality above as
%
\begin{equation*}
    R_T \leq \bbE_{X_0 \sim \cD} \sbr*{\wt R_T \spr*{X_0}} + 2\,\biasmgr.
\end{equation*}
%
We bound the two terms separately. In Lemma~\ref{lem::boundBias} we prove that the bias induced by MGR is upper bounded as
%
\begin{equation*}
    2 \biasmgr \leq \frac{\pi^2}{3}\,.
\end{equation*}
%
Thus, it remains to upper bound $\bbE_{X_0 \sim \cD} \sbr*{\wt R_T \spr*{X_0}}$. Let us fix a context $x \in \cX$ and recall that at time $t$, the distribution $q_t \spr*{\cdot \given x}$ is
%
\begin{equation*}
    q_t \spr*{\cdot \given x} = \argmin_{q \in \Delta_K} \scbr*{\sum_{s=1}^{t-1} \sumaK q \spr*{a} \inp*{x, \wt{\theta}_{s, a}} + \cR_t \spr*{q}},
\end{equation*}
%
where we denoted $\cR_t \spr*{q} = \frac{1}{\eta_t} H_\alpha \spr*{q} + \bar\beta H_{\bar\alpha} \spr*{q}$. By the standard FTRL analysis \citep[Exercise~28.12]{BanditBook} \antoine{copy/prove the lemma in Technical tools appendix and refer to it}, we have
%
\db{I guess for later we should write everything with expectations here? or bound $\bE[\stabt]$ below.}
\begin{align*}
    \wt R_T \spr*{x} &\leq \sumtT \underbrace{\spr*{\sumaK \spr*{q_t \spr*{a \given x} - q_{t+1} \spr*{a \given x}} \inp{x, \wt \theta_{t, a}} - D_{\cR_t} \spr*{q_{t+1} \spr*{\cdot \given x}, q_t \spr*{\cdot \given x}}}}_{= \stabt}\\
    &\quad+ \underbrace{\cR_{T+1} \spr*{\pi_T^\star \spr*{\cdot \given x}} - \cR_1 \spr*{q_1 \spr*{\cdot \given x}} + \sumtT \spr*{\frac{1}{\eta_{t+1}} - \frac{1}{\eta_t}} H_{\alpha} \spr*{q_{t+1} \spr*{\cdot \given x}}}_{= \pen}\\
    &\quad+ c K \sumtT p_t \spr*{x},
\end{align*}
%
where the Bregman divergence of the composite regularizer is given by
%
\begin{equation*}
    D_{\cR_t} \spr*{q, q'} = \cR_t \spr*{q} - \cR_t \spr*{q'} - \inp*{\nabla \cR_t \spr*{q'}, q - q'} = \frac{1}{\eta_t} D_{H_\alpha} \spr*{q, q'} + \bar\beta D_{H_{\bar\alpha}} \spr*{q, q'},
\end{equation*}
%
and the Bregman divergence associated to $H_\alpha$ is
%
\begin{align*}
    D_{H_\alpha} \spr*{q, q'} &= H_\alpha \spr*{q} - H_\alpha \spr*{q'} - \inp*{\nabla H_\alpha \spr*{q'}, q - q'}\\
    &= \frac{1}{\alpha} \sumaK \spr*{q_a - q_a' - q_a^\alpha + \spr*{q_a'}^\alpha + \spr*{\alpha \spr*{q_a'}^{\alpha - 1} - 1} \spr*{q_a - q_a'}}\\
    &= \frac{1}{\alpha} \sumaK \spr*{\spr*{q_a'}^\alpha - q_a^\alpha + \alpha \spr*{q_a'}^{\alpha - 1} \spr*{q_a - q_a'}}.
\end{align*}
\db{use the right entropy}

\paragraph{Controlling the stability term.} Denote $\stab = \sumtT \stabt$. Noting that $D_{H_{\bar\alpha}}$ is non-negative, we have
%
\begin{align*}
    \stabt &\leq \sumaK \spr*{q_t \spr*{a \given x} - q_{t+1} \spr*{a \given x}} \inp{x, \wt \theta_{t, a}} - \frac{1}{\eta_t} D_{H_\alpha} \spr*{q_{t+1} \spr*{\cdot \given x}, q_t \spr*{\cdot \given x}}\\
    &= \frac{1}{\eta_t} \spr*{\sumaK \spr*{q_t \spr*{a \given x} - q_{t+1} \spr*{a \given x}} \eta_t \inp{x, \wt \theta_{t, a}} - D_{H_\alpha} \spr*{q_{t+1} \spr*{\cdot \given x}, q_t \spr*{\cdot \given x}}}.
\end{align*}
%
We need to control the magnitude of $\eta_t \inp{x, \wt \theta_{t, a}}$. Using the definition of $\wt \theta_{t, a}$ and the fact that $\norm{x}_2 \leq 1$, we have \antoine{seems to be the first constraint on the choice of $\eta_t$, we need to set it so we can use Lemma~\ref{lem::ub_ftrl_technical}}
\db{Not on the choice of $\eta_t$ but on the choice of $u_t$. }
To be able to use Lemma~\ref{lem::ub_ftrl_technical}, we need to choose $u_t$ in order to verify that 
%
\begin{equation}\label{eq::as_bound}
    \abs{\eta_t \inp{x, \wt \theta_{t, a}}} \leq \frac{1 - \alpha}{4} \cdot \frac{1}{q_{t\star}(x)^{1 - \alpha}}
\end{equation}
For the moment, forget about MGR and assume that the estimate is of the form $\wt \theta_{t,a} = \Sigma_t^{-1}X_t\ell_t(X_t, a)\ind(a\in O_t)$, where $\Sigma_t$ is any matrix that makes the estimator unbiased/with controlled bias. Then, following e.g the proof of Lemma 2 of \cite{BOBWlinear} we can write that 
\begin{align*}
	\abs{\inp{x, \wt \theta_{t,a}}} &\leq \norm{\Sigma_t^{-1}}_{\text{op}} \\
	& \leq  \frac{1}{\lambda_{\min}(\Sigma_t)} \;, \\
\end{align*}
using that all contexts are in the unit Euclidean ball. Some examples are
\begin{itemize}
	\item[(i)]  $\Sigma_t=p_t(X_t) \Sigma$, which yields $\lambda_{\min}(\Sigma_t) = p_t(X_t)\cdot \lambda_{\min}(\Sigma)$.
	\item[(ii)] $\Sigma_t=\E_{t, X\sim \cD}\left[p_t(X) XX^T\right]$, which gives $\Sigma_t\succeq \lambda_{\min}(\Sigma)\; \underset{x\in \cX}{ \min } \;p_t(x)\cdot  I$ so $\lambda_{\min}(\Sigma_t) \geq \underset{x\in \cX}{ \min }\; p_t(x)  \cdot \lambda_{\min}(\Sigma)$.
	\item[(iii)] MGR estimate of (ii), for which it is known (\cite{BOBWlinear}[Lemma 2]) that $\norm{\Sigma_t^+}_{\text{op}} \leq \frac{(M_t+1)}{2}$.
\end{itemize}

The bound for the MGR estimates encourages to tune $M_t$ at most of order $\frac{1}{\eta_t}$, which is what is done in \cite{BOBWlinear}. This can in turn impose constraints on the tuning of $\eta_t$ (in that case the tuning of $u_t$ will appear in the control of the bias), which gives the extra logs in \cite{BOBWlinear}.  

Apart from the MGR estimate, it seems that the above bounds impose a strong constraint on $p_t$, and so on $u_t$. Indeed, as in \cite{BOBWhardproblems} we use $u_t$ to meet the condition above by choosing it to be $u_t(x)=\eta_t v_t(x)$ for some function $v$. Unfortunately, without proving more properties on $p_t$ it seems that $u_t$ must be uniformly bounded (in both case). In particular, it seems that one must impose 
\begin{equation}\label{eq::cond_ut}\forall x \in \cX, \; u_t(x) = \frac{4\eta_t}{1-\alpha} \cdot \max_{x\in \cX} q_{t\star}(x)^{1-\alpha} \;. 
\end{equation}

\db{This is a bit hard to sell from a computational perspective. Can we find less ugly? If we replace the prob by $1/2$ what do we lose? Do we still get BOBW but get uglier logs? Note that if the context space is somewhat continuous nothing prevent several arms to be optimal/close to optimal, so this would be constant anyway\dots}

\db{An alternative path, but maybe super hard would be to upper bound the ratio $\sup_{x,y} \frac{q_{t, \star}(x)}{q_{t,\star}(y)}$.}

\db{Alternative alternative path: we can use the MGR samples to estimate the map of observation probabilities over contexts. Maybe we can use some sort of smoothness arguments to show that this would be enough to get a good tuning for $u_t$?}


If we can guarantee that \eqref{eq::as_bound} holds, we can then use Lemma~\ref{lem::ub_ftrl_technical} to bound the stability term by 

\[\stab\leq \frac{4}{1-\alpha} \bE\left[\sum_{t=1}^T   \bE_t\left[ \sum_{i \neq \wt I_t} q_{t,i}(x)^{2-\alpha} \wt \ell_{t,i}(x)^2 + q_{t\star}(x)^{2-\alpha} \wt \ell_{t,\wt I_t}(x)^2 \right] \right]\;. \]

To fit in the framework of \cite{BOBWhardproblems} we need to upper bound this term by $\frac{z_t(x)}{\beta_t \gamma'_t(x)}=\sqrt{\frac{z_t(x)}{\beta_t}}$.

Consider first estimate (i). Following Eq.\;(8) from \cite{NeuO20} we can use that, for all $i \in [K]$,  
\begin{align*}
\bE_t\left[\wt \ell_{t,i}^2\right]&\leq\bE_t\left[\frac{\ind(a\in O_t)}{p_t(X_t)^2} x^\top\Sigma^{-1}X_tX_t^\top\Sigma^{-1}x\right] \\
& = \bE_t\left[\frac{\ind(a\in O_t)}{p_t(X_t)^2}\text{Tr}(\Sigma^{-1}x x^\top\Sigma^{-1}X_t X_t^\top)\right] \\
& \leq \frac{d}{p_t(X_t)} \;.
\end{align*}
\db{Not usable unless we use the same observation probability for everyone, and get a unique $p_t=\sqrt{\frac{\bE_{X\sim \cD}[z_t(X)]}{\beta_t}}$. Let us elaborate a bit.}
Let us introduce 
\[Z_t(x)\coloneqq   \sum_{i \neq \wt I_t} q_{t,i}(x)^{2-\alpha} + q_{t\star}(x)^{2-\alpha}\;. \]
Then, if we fix a shared $p(X_t)=\bar p_t$ for all $t$ and $X_t$, we can obtain that 
\[\bE_t[\stabt] \leq \beta_t^{-1} \underbrace{(cK \bar p_t)^{-1}}_{\gamma_t^{-1}} \cdot  \underbrace{\frac{4cKd}{1-\alpha}\cdot \bE_{X\sim \cD}[Z_t(X)]}_{z_t} , \] 
if we consider that the ghost sample is independent from the filtration (that is, the filtration is just about things that were seen by the alg). With this writing, the $z_t$ that we fit in the framework of \cite{BOBWhardproblems} is defined as an expectation over contexts, and the SPB matching can then proceed as they describe. \db{Which needs to be checked though, in particular the next step is to have a look at the entropy property for the penalty term.}

\db{All of this is Ok if we can evaluate it up to a good approximation.} \db{So in summary, with estimate (i) it seems that we can complete the bound on the stability term using the SPB matching definition, and defining $u_t$ as in Eq.~\eqref{eq::cond_ut}. Note that in the case where $p_t$ is uniform then estimates (i) and (ii) match and MGR is not necessary, or just before the interaction to get an approximation of $\Sigma^{-1}$ up to an arbitrary precision.}


Using $\Sigma_t$ as in (ii), we get 
\begin{align*}
	\bE_t\left[\wt \ell_{t,i}^2\right]&\leq\bE_t\left[\ind(a\in O_t) x^\top\Sigma_t^{-1}X_tX_t^\top\Sigma_t^{-1}x\right] \\ &  =
	 x^\top\Sigma_t^{-1}\cdot p_t(X_t)X_tX_t^\top\Sigma_t^{-1}x \;.
\end{align*}
We then write that 
\begin{align}
	\stab &\leq \frac{4}{\beta_t(1-\alpha)} \sum_{t=1}^T \bE\left[\frac{Z_t(x)}{p_t(x)} \text{Tr}(p_t(x)xx^\top \Sigma_t^{-1} X_t X_t^\top p_t(X_t)\Sigma_t^{-1})\right] \;.
\end{align}
Now to be able to do the same as for Lemma 6 of \cite{NeuO20} we would need to have $Z_t(x)/p_t(x)=1$, so the expectation could go in the trace and yield only a $d$ factor. \db{I don't know how to deal with this term, which looks rather necessary if we don't want to use a uniform probability.}  


\paragraph{Controlling the penalty term.} \db{Previous draft} \antoine{If I understand correctly the second regularizer becomes useful to enable the use of Lemma~\ref{lem::ftrl_smooth_bound}}
\db{The second regularizer controls the ratio of entropies in-between rounds, that is condition (iii) in Thm. 7 of \cite{BOBWhardproblems}.}

\db{Since the learning rate update is defined for SPB matching the only thing to do here is to control this entropy. Seems all arguments from \cite{BOBWhardproblems} should readily apply here with the ghost sample trick.}

\db{Tldr, this part should be trivial with context-dependent probs.}

\paragraph{Penalty term under $\bar p_t$ algorithm design}


%%%%%%%%%%%%%%%%%%%%%%%%%%%%%%%%%%%%%%%%%%%%%%%%%%%%%%%%%%%%%%%%%%%
%%%%%%%%%%%%%%%%%%%%%%%%%%%%%%%%%%%%%%%%%%%%%%%%%%%%%%%%%%%%%%%%%%%
\section{PROOF OF LEMMA~\ref{lem::adapt_tsuchaya}} \label{Appendix}

\adath*

\begin{proof}[Proof of Lemma~\ref{lem::adapt_tsuchaya}]
The argument follows the same general structure of the proof of Theorem~7 in \cite{BOBWhardproblems}. 
We first define
    \begin{equation*}
\gamma'_t \coloneqq \gamma_t - \frac{u_t}{\beta_t} = \sqrt{\frac{z_t}{\beta_t}}.
    \end{equation*}


Starting from the regret decomposition given by Assumption~(i) of the lemma, we have:
\begin{align*}
R_T 
&\leq \bbE\Bigg[\sumtT \Big( 
\Big(\frac{1}{\eta_t} - \frac{1}{\eta_{t-1}}\Big) h_t 
+ \frac{z_t \eta_t}{\gamma_t}
+ \gamma_t \Big)  \Bigg] + \bar{\beta}\bar{h}
\\
&\text{\small (by Assumption (i) of the lemma)} \\[0.4em]
&= \bbE\Bigg[\sumtT \Big( 
(\beta_t - \beta_{t-1}) h_t 
+ \frac{z_t \eta_t}{\gamma_t} 
+ \gamma_t \Big) \Bigg]+ \bar{\beta}\bar{h}
\end{align*}

where we used $\beta_t = 1/\eta_t$ and $\bar{h} = \max_{p \in \Delta_K} H_{\bar{\alpha}}(p)$.
We now replace $\gamma_t$ by $\gamma'_t \le \gamma_t$ to simplify the analysis (this may loosen the bound slightly but keeps the algebra tractable):
\begin{align*}
R_T 
&\leq \bbE\Bigg[\sumtT \Big( 
(\beta_t - \beta_{t-1}) h_t 
+ \frac{z_t \eta_t}{\gamma'_t} 
+ \gamma_t \Big) \Bigg]+ \bar{\beta}\bar{h} \\[0.4em]
&= \sumtT \bbE\!\left[ 
(\beta_t - \beta_{t-1}) h_t 
+ \frac{z_t \eta_t}{\gamma'_t} 
+ \gamma_t \right]
+ \bar{\beta}\bar{h}
\end{align*}

\paragraph{Bounding the first term.}
We first upper bound $\sumtT \bbE[(\beta_t - \beta_{t-1}) h_t]$.  
By the tower rule and the fact that $(\beta_t - \beta_{t-1})$ is $\cH_{t-1}$-measurable, we have
\begin{align*}
\sumtT \bbE \!\left[ (\beta_t - \beta_{t-1}) h_t \right] 
&= \sumtT \bbE \!\left[ 
(\beta_t - \beta_{t-1}) \, 
\bbE[h_t \mid \cH_{t-1}] \right].
\end{align*}
Then, by Assumption~(ii), which ensures 
$\bbE[h_t \mid \cH_{t-1}] \le 2 \, \bbE[h_{t-1} \mid \cH_{t-2}]$, we get:
\begin{align*}
\sumtT \bbE \!\left[ (\beta_t - \beta_{t-1}) h_t \right]
&\le 2 \sumtT \bbE\!\left[ (\beta_t - \beta_{t-1}) h_{t-1} \right].
\end{align*}

\paragraph{Bounding the remaining terms.}
Using the definitions of $\gamma_t$ and $\gamma_t'$, namely
$\gamma'_t = \sqrt{z_t / \beta_t}$ and $\gamma_t = \gamma'_t + u_t / \beta_t$, we have
\begin{align}\label{eq::critical step}
\sumtT \bbE\!\left[\frac{z_t \eta_t}{\gamma'_t} + \gamma_t \right]
&= \sumtT \bbE\!\left[
\sqrt{\frac{z_t}{\beta_t}} + 
\Big( \sqrt{\frac{z_t}{\beta_t}} + \frac{u_t}{\beta_t} \Big) \right] \\
&\le \sumtT \bbE\!\left[
2\sqrt{\frac{z_t}{\beta_t}} + \frac{u_t}{\beta_t} \right]. \nonumber
\end{align}

Combining the two results above, we obtain:
    \begin{equation*}
R_T \le 
\bbE\!\left[
F(\beta_{1:T}, z_{1:T}, u_{1:T}, h_{0:T-1})
\right]
+ \bbE[\bar{\beta}\bar{h}],
    \end{equation*}
where we define
    \begin{equation*}
F(\beta_{1:T}, z_{1:T}, u_{1:T}, h_{1:T})
\coloneqq \sumtT 
\left(
(\beta_t - \beta_{t-1}) h_t
+ 2\sqrt{\frac{z_t}{\beta_t}}
+ \frac{u_t}{\beta_t}
\right).
    \end{equation*}
Note that the regret upper bound we obtained at this step involves the sequence $h_{0:T-1}$, and not $h_{1:T}$ as in the above definition.  


\paragraph{Adversarial regime}
Using Lemma~\ref{hardthm6},
we obtain that, for any $\epsilon \ge 1/T$,
\begin{align*}
F(\beta_{1:T}, z_{1:T}, u_{1:T}, h_{0:T-1})
&\le \left( 
\left[ \sumtT \sqrt{z_t h_t} \right] 
\log(\epsilon T)
\right)^{2/3} \\
&\quad + \sqrt{
\left[ \sumtT u_t h_t \right]
\log(\epsilon T)
} \\
&\quad + 
\left( \frac{\sqrt{z_{\max} h_1}}{\epsilon} \right)^{2/3}
+ \sqrt{ \frac{u_{\max} h_1}{\epsilon} }
+ \kappa.
\end{align*}

Substituting this into the previous inequality gives the claimed regret bound for the adversarial case:
\begin{align*}
R_T \le &
\left( 
\bbE\left[\sumtT \sqrt{z_t h_t}\right]
\log(\epsilon T)
\right)^{2/3} 
+ 
\sqrt{
\bbE\left[\sumtT u_t h_t\right]
\log(\epsilon T)
} \\
& + 
\left( \frac{\sqrt{z_{\max} h_1}}{\epsilon} \right)^{2/3}
+ 
\sqrt{ \frac{u_{\max} h_1}{\epsilon} }
+ \kappa.
\end{align*}
Setting $\epsilon = 1/T$ yields the desired bound in the adversarial regime.

\paragraph{Stochastic regime.}
We now turn to the stochastic case, under Assumptions~(iii)–(iv). 
Define
    \begin{equation*}
\varrho_0(\pi^\star_T) 
\coloneqq \sumtT 
\left( 1 - q_t(\pi^\star_T(X_t) \mid X_t) \right).
    \end{equation*}
%\red{$\pi^\star$ should be replaced by $\pi^\star_T$ below}
By Assumptions~(iii)–(iv),
\begin{align*}
\bbE\!\left[\sumtT \sqrt{z_t h_t}\right]
&\le \sqrt{\rho} \cdot \varrho_0(\pi_T^\star), \\
\bbE\!\left[\sumtT u_t h_t\right]
&\le \rho \cdot \varrho_0(\pi_T^\star).
\end{align*}
Furthermore, Lemma~21 of \cite{BOBWlinear} gives the lower bound
    \begin{equation*}
R_T \ge \frac{\deltamin}{2} \, \bbE[\varrho_0(\pi^\star)] - 2C.
    \end{equation*}

Balancing both bounds using any $\lambda \in (0,1]$, and applying the inequalities 
$a x^2 - b x^3 \le \tfrac{4 a^3}{27 b^2}$ 
and 
$a x - b x^2 \le \tfrac{a^2}{4b}$ 
(for $a \ge 0$, $b > 0$), 
we obtain after simplification:
\begin{align*}
R_T \lesssim
&\frac{(1+\lambda)^3}{\lambda^2} 
\cdot \frac{\rho \log(\epsilon T)}{\deltamin^2}
+ \frac{(1+\lambda)^2}{\lambda} 
\cdot \frac{\rho \log(\epsilon T)}{\deltamin} \\
&\quad + 
\left( \frac{\sqrt{z_{\max} h_1}}{\epsilon} \right)^{2/3}
+ \sqrt{ \frac{u_{\max} h_1}{\epsilon} }
+ \kappa + 2\lambda C.
\end{align*}

Choosing $\lambda = \Theta\!\left( 
\left( \tfrac{\rho \log(\epsilon T)}{C} \right)^{1/3}
\right)$ 
and setting $\epsilon = 1 / (\rho^2/\deltamin^3 + C\rho/\deltamin) \le 1/T$
gives, for $T \ge \tau \coloneqq \tfrac{1}{\deltamin^3} + \tfrac{C}{\deltamin}$,
\begin{align*}
R_T \lesssim
&\frac{\rho}{\deltamin^2} 
\log_+\!\left( T\deltamin^3 \right)
+ 
\left(
\frac{C^2 \rho}{\deltamin^2}
\log_+\!\left( \frac{T\deltamin}{C} \right)
\right)^{1/3} \\
&+ 
\left( (z_{\max} h_1)^{1/3}
+ \sqrt{u_{\max} h_1} \right)
\left( \frac{1}{\deltamin^3} + \frac{C}{\deltamin} \right)^{2/3}
+ \kappa,
\end{align*}
which concludes the proof.
\end{proof}



%%%%%%%%%%%%%%%%%%%%%%%%%%%%%%%%%%%%%%%%%%%%%%%%%%%%%%%%%%%%%%%%%%%
%%%%%%%%%%%%%%%%%%%%%%%%%%%%%%%%%%%%%%%%%%%%%%%%%%%%%%%%%%%%%%%%%%%
\section{PROOF OF THEOREM~\ref{thm::main}}\label{AppendixRegret}


We build on Lemma~\ref{lem::adapt_tsuchaya}, presented and proved in Appendix~\ref{Appendix}, to prove Theorem~\ref{thm::main} by verifying that Algorithm~\ref{alg::FTRL_bobw} satisfies conditions~(i)–(iv) of the lemma. We recall the theorem below, before presenting its proof. 

\MainTheorem*


\begin{proof} \DB{Tldr: all this part is shit because it mixes using fixed context and current context, with ambiguous notation that doesn't allow to understand in which case we are.}
We verify that Algorithm~\ref{alg::FTRL_bobw} satisfies the four conditions of Lemma~\ref{lem::adapt_tsuchaya}.



Throughout this proof, we work with the \emph{exact} loss estimates $\wh{\theta}_{t, a}$ defined in Eq.~\eqref{eq::estimator}, rather than their MGR approximations $\wt{\theta}_{t, a}$ used in the algorithmic description. 
This distinction is only technical and does not affect the regret order, since Lemma~\ref{lem::boundBias} guarantees that the cumulative bias introduced by the MGR approximation remains uniformly bounded.

\noindent\textbf{Condition (i).} 
By definition of the importance-sampled loss (Eq.~\eqref{eq::estimator}), 
for any $a \in \sbr*{K}$ we have
    \begin{equation*}
|\wh{\ell}_{t, a} \eta_t|
\le \frac{\ell_{t, a}\eta_t}{p_t \lambda_{\min}}
\le \frac{1}{u_t \lambda_{\min}}
\le \frac{1 - \alpha}{8} 
\cdot \frac{1}{\min(q_{t,a_t^\star}, 1 - q_{t,a_t^\star})^{1 - \alpha}}.
    \end{equation*}
Hence, the scaled losses $\wh{\ell}_t \eta_t$ satisfy the condition of Lemma~\ref{lem::ub_ftrl_technical}, presented in Appendix~\ref{Appendix}, which provides an upper bound on the penalty term 
\(\langle q_t - q_{t+1}, \wh{\ell}_t \spr*{x}\rangle - D_t(q_{t+1},q_t)\) 
appearing in the standard FTRL regret decomposition.

\DB{Imo, has to be re-written from the start with the ghost sample $x$ and with the estimates $\wt \theta_{t, a}$ actually used by FTRL.}
Since the regret is defined by 
    \begin{equation*}
R_T = 
\bbE\!\left[ 
\sumtT 
\left( 
\langle X_t, \theta_{t,A_t} \rangle 
- \langle X_t, \theta_{t,\pi^\star(X_t)} \rangle 
\right) 
+ cK \sumtT p_t
\right],
    \end{equation*}
and $\wh{\theta}_{t, a}$ is an unbiased estimator of $\theta_{t, a}$, we can equivalently write
    \begin{equation*}
R_T =
\bbE\!\left[
\sumtT 
\left(
\langle X_t, \wh{\theta}_{t,A_t} \rangle 
- \langle X_t, \wh{\theta}_{t,\pi^\star(X_t)} \rangle 
\right)
+ cK \sumtT p_t
\right].
    \end{equation*}
\DB{This is wrong without using the ghost sample technique, cause the estimates depend on $X_t$ (with some $X_0$, it's OK).}

Fix any context $x \in \bbR^d$.  
Applying Lemma~\ref{lem::standard_regret_decompo}, we obtain:
\begin{align*}
\sumtT 
\left( 
\langle x, \wh{\theta}_{t,A_t} \rangle 
- \langle x, \wh{\theta}_{t,\pi^\star \spr*{x}} \rangle 
\right)
\le
&\quad\underbrace{
\sumtT \big( \psi_t(q_{t+1}(.|x)) - \psi_{t+1}(q_{t+1}(.|x)) \big)
}_{\text{penalty}} \\[0.3em]
&+ 
\underbrace{
\sumtT \big( 
\langle q_t(.|x) - q_{t+1}(.|x), \wh{\ell}_t \spr*{x} \rangle 
- D_t(q_{t+1}(.|x), q_t(.|x))
\big)
}_{\text{stability}} 
+ A + \bar{\beta}\bar{h},
\end{align*}
where $A = \psi_{T+1}(\pi^\star(\cdot|x)) - \psi_1(q_1(\cdot|x)) \le \beta_1 \log K$ is independent of $T$ and will be ignored in the sequel (together with $\bar{\beta}\bar{h}$).

\paragraph{Bounding the penalty term.}
By the definition of $\psi_t$, we have
    \begin{equation*}
\sumtT 
\big( \psi_t(q_{t+1}(.|x)) - \psi_{t+1}(q_{t+1}(.|x)) \big)
\le 
\sumtT 
\Big( \frac{1}{\eta_{t+1}} - \frac{1}{\eta_t} \Big) h_{t+1}.
    \end{equation*}
\DB{Now here this is were messing up with the notation is confusing: above check which components should depend on $x$.}
Reindexing $t \mapsto t-1$ yields the equivalent form
    \begin{equation*}
\sumtT 
\Big( \frac{1}{\eta_t} - \frac{1}{\eta_{t-1}} \Big) h_t.
    \end{equation*}

\paragraph{Bounding the stability term.}
Using Lemma~\ref{lem::ub_ftrl_technical} together with Lemma~\ref{lem::bound_squared_loss}, we have
\begin{align}
\sumtT 
\big(
\langle q_t - q_{t+1}, \wh{\ell}_t \spr*{x} \rangle 
- D_t(q_{t+1}, q_t)
\big) 
&= 
\sumtT 
\frac{1}{\eta_t}
\left(
\langle q_t - q_{t+1}, \wh{\ell}_t \spr*{x}\eta_t \rangle
- D_t(q_{t+1}, q_t)
\right) \nonumber\\
&\le
\sumtT 
\frac{4 \eta_t}{1 - \alpha}
\left(
q_{t,a_t^\star}^{2 - \alpha} \wh{\ell}_{t, a_t^\star}^2
+ \sum_{a \neq a_t^\star} q_{t, a}^{2 - \alpha} \wh{\ell}_{t, a}^2
\right) \label{eq::bad_conditioning} \\
&\le
\sumtT 
\frac{4 d^2 \eta_t}{p_t (1 - \alpha) \lambda_{\min}^2}
\left(
q_{t,a_t^\star}^{2 - \alpha}
+ \sum_{a \neq a_t^\star} q_{t, a}^{2 - \alpha}
\right). \nonumber
\end{align}
\DB{Isn't there some $\min(q,1-q)$ at some point in lemma 3? Otherwise why split between $a_t^\star$ and $a\neq a_t^\star$. Not consistent}

\DB{More important: lemma 2 bounds $\bE[\widehat l_{t, a}^2]$ while here you should have a bound on $\bE[q_{t, a}^{2-\alpha} \cdot \widehat\ell_{t, a}^2]$. Transposing previous works we would work on $\bE[\sum_{a=1}^K q_{t, a}^{2-\alpha} \cdot \widehat\ell_{t, a}^2]$ directly.}

Taking expectations over the random context $X_t$, we obtain
\begin{align*}
\bbE[R_T]
&\le 
\bbE\Bigg[
\sumtT 
\Big( \frac{1}{\eta_t} - \frac{1}{\eta_{t-1}} \Big) h_t \\
&\quad +
\sumtT 
\frac{4 d^2 \eta_t}{p_t (1 - \alpha) \lambda_{\min}^2}
\Big(
q_{t,a_t^\star}^{2 - \alpha}
+ \sum_{a \neq a_t^\star} q_{t, a}^{2 - \alpha}
\Big)
+ cK \sumtT p_t
\Bigg],
\end{align*}
which matches the required structure of condition~(i).

\DB{It seems we should not be doing that but instead use a ghost sample to define the regret as in \cite{BOBWhardproblems, BOBWlinear}.}

\paragraph{Condition (ii).}
Condition~(ii) follows directly from Lemma~\ref{lem::cond_bound_ht}, presented and proved in Appendix~\ref{Appendix}, which guarantees that
    \begin{equation*}
\bbE[h_{t+1} \mid \cH_t]
\le 
2 \, \bbE[h_t \mid \cH_{t-1}],
\qquad \forall t \ge 1.
    \end{equation*}

\paragraph{Conditions (iii) and (iv).}
Lemma~13 of \cite{BOBWhardproblems} provides an upper bound on the entropy term,
    \begin{equation*}
h_t \le \frac{1}{\alpha}(K-1)^{1-\alpha} (1 - q_{t,a_t^\star})^{\alpha},
    \end{equation*}
where $a_t^\star \coloneqq \text{arg max}_{a \in \sbr*{K}} \langle X_t, \theta_{t, a} \rangle$ denotes the optimal arm for context $X_t$.
Moreover, using the definitions of $z_t$ and $u_t$, we obtain:
\begin{align*}
z_t
&= \frac{4 c K d^2}{(1 - \alpha) \lambda_{\min}^2}
\left(
\sum_{a \neq a_t^\star} q_{t, a}^{2 - \alpha}
+ (\min(q_{t,a_t^\star}, 1 - q_{t,a_t^\star}))^{2 - \alpha}
\right) \\
&\le 
\frac{8 c K d^2}{(1 - \alpha) \lambda_{\min}^2}
(1 - q_{t,a_t^\star})^{2 - \alpha}.
\end{align*}
Combining the bounds on $h_t$ and $z_t$ yields
\begin{align*}
z_t h_t 
&\le 
\frac{8 c K d^2 (K-1)^{1-\alpha}}{\alpha\spr*{1 - \alpha}\lambda_{\min}^2}
(1 - q_{t,a_t^\star})^2,\\
u_t h_t 
&\le 
\frac{8 d \max(c,1)}{\spr*{1 - \alpha}\alpha}(K-1)^{1-\alpha}
(1 - q_{t,a_t^\star}).
\end{align*}
Hence, both conditions are satisfied with
\begin{align*}
\sqrt{z_t h_t} &\le \sqrt{\rho}\,(1 - q_{t,a_t^\star}),\\
u_t h_t &\le \rho\,(1 - q_{t,a_t^\star}),
\end{align*}
where
    \begin{equation*}
\rho \coloneqq 
\frac{d}{\lambda_{\min}}
\max\!\left(
\sqrt{\frac{8cK(K-1)^{1-\alpha}}{\alpha\spr*{1 - \alpha}}},
\frac{8 \max(c,1)}{\spr*{1 - \alpha}\alpha}(K-1)^{1-\alpha}
\right).
    \end{equation*}

\paragraph{Conclusion.}
Having verified conditions~(i)–(iv), we can invoke Lemma~\ref{lem::adapt_tsuchaya} to conclude that Algorithm~\ref{alg::FTRL_bobw} enjoys a Best-of-Both-Worlds (BoBW) regret guarantee.  
To make the constants explicit, note that
    \begin{equation*}
h_{\max} \le \frac{K^{1-\alpha}}{\alpha}, \qquad
z_{\max} = \mathcal{O}\!\left(\frac{cK d^2}{(1 - \alpha)\lambda_{\min}^2}\right), \qquad
u_{\max} = \mathcal{O}\!\left(\frac{d \max(c,1)}{1 - \alpha}\right).
    \end{equation*}
Plugging these into Lemma~\ref{lem::adapt_tsuchaya} gives
\begin{align*}
\text{Adversarial regime: } &
R_T = \mathcal{O}\!\left(
\left( \frac{cK d^2}{\lambda_{\min}^2} \right)^{1/3} T^{2/3}
+ \sqrt{ \frac{d \max(c,1) T}{\lambda_{\min}} }
\right),\\
\text{Corrupted stochastic regime: } &
R_T = \mathcal{O}\!\left(
\frac{d \sqrt{\max(c,1)K}}{\lambda_{\min}\deltamin^2}\log(T\deltamin^3)
+ \left(
\frac{C^2 d \sqrt{\max(c,1)K}}{\lambda_{\min}\deltamin^2}
\log\!\frac{T\deltamin}{C}
\right)^{1/3}
\right).
\end{align*}

This completes the proof of Theorem~\ref{thm::main}.
\end{proof}



%%%%%%%%%%%%%%%%%%%%%%%%%%%%%%%%%%%%%%%%%%%%%%%%%%%%%%%%%%%%%%%%%%%
%%%%%%%%%%%%%%%%%%%%%%%%%%%%%%%%%%%%%%%%%%%%%%%%%%%%%%%%%%%%%%%%%%%

\clearpage
\section{Technical tools}

\begin{lemma}[\citealp{NeuO20}, Lemma~3] \label{lem:ghost-sample-neu}
    Let $\pi^\star$ be a fixed stochastic policy and let $X_0$ be a sample from the context distribution $\cD$ independent from $\cH_T$. For any $t \in \sbr*{T}$, any action $a \in \sbr*{K}$, suppose that $\pi_t$ is $\cH_{t - 1}$-measurable and that $\bbE \sbr*{\wh{\theta}_{t, a} \given \cH_{t - 1}} = \theta_{t, a}$. Then, it holds that
    %
    \begin{equation*}
        \bbE \sbr*{\sumtT \sumaK \spr*{\pi_t \spr*{a \given X_t} - \pi^\star \spr*{a \given X_t}} \inp*{X_t, \theta_{t, a}}} = \bbE \sbr*{\sumtT \sumaK \spr*{\pi_t \spr*{a \given X_0} - \pi^\star \spr*{a \given X_0}} \inp*{X_0, \wh{\theta}_{t, a}}}.
    \end{equation*}
\end{lemma}

\begin{lemma} \label{lem::ftrl_regret_decomp}
    \antoine{Exercise 28.12 from \citet{BanditBook} as a lemma + proof}
\end{lemma}

\begin{lemma}
    \antoine{Convexity of the negative Tsallis entropy, corresponding Bregman divergence}
\end{lemma}

\antoine{Some of the lemmas below should stay in this section, but the most important ones should probably move to the appendix ``Omitted Proofs''.}

\DB{Below we have a problem, $\lambda_{\min}(\Sigma_{t, a}) \ge p_t \, \lambda_{\min}$ has no reason to hold and this should be fixed.}

\DB{Additionally, we haven't sorted the dimension issue yet. What should be the right dependence in $d$? It's not so clear, but it should appear in the squared loss if we follow previous works: there, $d$ come from using a trace bound instead of our inequalities below. Some of our steps are probably wrong.}

\DB{The following result should bound $\bE[\sum_{a} q_{t, a}^{2-\alpha} \widehat \ell_{t, a}^2]$ or whatever is needed in the part of the proof where it's used.}
\begin{lemma}\label{lem::bound_squared_loss}
Let $X_t \in \bbR^d$ be a random context and fix any arm $a \in \sbr*{K}$. 

Under the assumptions of Section~\ref{sec::setting}, we have $\|X_t\|_2 \le 1$ almost surely, and the loss function satisfies $-1 \le \ell_t \spr*{X_t, a} \le 1$.

We also recall that $\Sigma_{t, a}$ is a positive definite matrix such that 
    \begin{equation*}
\lambda_{\min}(\Sigma_{t, a}) \ge p_t \, \lambda_{\min},
    \end{equation*}
\DB{We should probably adopt a notation $p_t \spr*{x}$, a lot of confusion comes from using $p_t$ for many different things. The above line is clearly wrong (if $p_t$ is $p_t(X_t)$). It is true with $p_{t, \min}\coloneqq \min_{x\in \cX}p_t \spr*{x}$ though, and could be replaced by $\frac{\gamma}{K}$ if we add a forced exploration $\gamma$.}
and that the importance-weighted estimator is given by
    \begin{equation*}
\wh{\theta}_{t, a} 
\coloneqq \Sigma_{t, a}^{-1} X_t \, \ell_t \spr*{X_t, a} \, \mathbf{1}\{a \in O_t\},
    \end{equation*}
where $\mathbb{P}(a \in O_t) = p_t$.
Then,
    \begin{equation*}
\bbE\!\left[\langle X_t, \wh{\theta}_{t, a} \rangle^2\right] 
\;\le\; \frac{1}{\lambda_{\min}^2 \, p_t}.
    \end{equation*}
\end{lemma}

\begin{proof}
We know that the smallest eigenvalue of $\Sigma_{t, a}$ is $\ge p_t \lambda_{\min}$. 
    \begin{equation*}
\|\Sigma_{t, a}^{-1}\|_2 
\;\le\; \frac{1}{\lambda_{\min}(\Sigma_{t, a})}
\;\le\; \frac{1}{p_t \lambda_{\min}}.
    \end{equation*}
Therefore,
    \begin{equation*}
\|\wh{\theta}_{t, a}\|_2
= \|\Sigma_{t, a}^{-1} X_t \, \ell_t \spr*{X_t, a}\, \mathbf{1}\{a \in O_t\}\|_2
\le \frac{\|X_t\|_2}{p_t \lambda_{\min}} \, \mathbf{1}\{a \in O_t\}.
    \end{equation*}
By the Cauchy–Schwarz inequality,
    \begin{equation*}
\langle X_t, \wh{\theta}_{t, a} \rangle^2
\le \|X_t\|_2^2 \, \|\wh{\theta}_{t, a}\|_2^2
\le \frac{\|X_t\|_2^4}{p_t^2 \lambda_{\min}^2} \, \mathbf{1}\{a \in O_t\}.
    \end{equation*}
Taking expectations and using $\bbE[\mathbf{1}\{a \in O_t\}] = p_t$, we obtain
    \begin{equation*}
\bbE\!\left[\langle X_t, \wh{\theta}_{t, a} \rangle^2\right]
\le \frac{\bbE[\|X_t\|_2^4]}{p_t \lambda_{\min}^2}.
    \end{equation*}
Since $\|X_t\|_2 \le \sqrt{d}$ almost surely, it follows that $\bbE[\|X_t\|_2^4] \le 1$, and hence
    \begin{equation*}
\bbE\!\left[\langle X_t, \wh{\theta}_{t, a} \rangle^2\right]
\le \frac{1}{\lambda_{\min}^2 \, p_t}.
    \end{equation*}
\end{proof}

\db{(minor for now) Write a few sentences between lemmas. Also, here we just do an unstructured list, but actually the above lemma is crucial, while the following is just a restatement of an existing result. Might be nice to structure this.}

\begin{lemma}[Lemma~14 in \cite{BOBWhardproblems}]\label{lem::ub_ftrl_technical}
Let $q \in \mathcal{P}_K$ and let $\bar{I} \in \arg\max_{i \in \sbr*{K}} q_i$.  
Let $l \in \bbR^K$ be such that, for all $i \in \sbr*{K}$,
    \begin{equation*}
|l_i| \leq \frac{1 - \alpha}{4} \cdot \frac{1}{\min(q_{\bar{I}}, 1 - q_{\bar{I}})^{1 - \alpha}}.
    \end{equation*}
Then, the following bound holds:
    \begin{equation*}
\max_{p \in \mathcal{P}_K} \left\{ 
\langle l, q - p \rangle - D_{-H_\alpha}(p, q) 
\right\}
\leq \frac{4}{1 - \alpha} \Bigg( 
\sum_{i \neq \bar{I}} q_i^{2 - \alpha} l_i^2
+ \min(q_{\bar{I}}, 1 - q_{\bar{I}})^{2 - \alpha} l_{\bar{I}}^2 
\Bigg)
    \end{equation*}

\end{lemma}


\begin{lemma}\label{lem::standard_regret_decompo}
Let $x\in\bbR^d$ be any fixed context. For each $t\ge1$, let
$q_t(\cdot|x)\in\Delta_K$ be the distribution used to sample $A_t$
given $x$, and let $\pi^{*}(\cdot|x)\in\Delta_K$ be any comparator
policy. Let $(\psi_t)_{t\ge1}$ be a sequence of
$\sigma$-strongly convex regularizers on $\Delta_K$, and let
$D_t(\cdot,\cdot)$ denote the Bregman divergence induced by $\psi_t$.
Denoting $\wh\ell_t \spr*{x}\in\bbR^K$ for the vector of estimated losses
at context $x$, with $[\wh\ell_t \spr*{x}]_a \coloneqq \langle x,\wh\theta_{t, a}\rangle$.
Then
\begin{align*}
\bbE\!\left[\sumtT 
\Big(\langle x,\wh\theta_{t,A_t}\rangle-\langle x,\wh\theta_{t,\pi^{*} \spr*{x}}\rangle\Big)\right]
&\le
\bbE\!\left[\sumtT \big(\psi_t(q_{t+1}(.|x))-\psi_{t+1}(q_{t+1}(.|x))\big)\right] \\
&+ \bbE\!\left[\sumtT \big(\langle q_t(.|x)-q_{t+1}(.|x),\wh\ell_t \spr*{x}\rangle - D_t(q_{t+1}(.|x),q_t(.|x))\big)\right] \\
&
+ \bbE\!\left[\psi_{T+1}(\pi^{*}(\cdot|x))-\psi_1(q_1(\cdot|x))\right]
+ \bar\beta\,\bar h,
\end{align*}
where $\bar h\coloneqq\max_{p\in\Delta_K}H_{\bar\alpha}(p)$ and $\bar\beta\ge0$
is the coefficient that upper-bounds the change of regularizer in our setting.
\end{lemma}

\DB{This lemma above uses a fixed context, so is suited for a proof that uses the ghost sample technique. It cannot work for the proof as it is right now.}

\DB{Has to be modified if forced exploration is introduced. See (44) in \cite{BOBWlinear} for their result.}

\begin{proof}
Conditionally on $x$, $A_t\sim q_t(\cdot|x)$, hence
    \begin{equation*}
\bbE\!\left[\langle x,\wh\theta_{t,A_t}\rangle \,\middle|\, x\right]
= \sum_{a \in \sbr*{K}} q_t(a|x)\,\langle x,\wh\theta_{t, a}\rangle.
    \end{equation*}
Therefore,
\begin{align*}
\bbE\!\left[\sumtT 
\big(\langle x,\wh\theta_{t,A_t}\rangle-\langle x,\wh\theta_{t,\pi^{*} \spr*{x}}\rangle\big)\right]
&=
\bbE\!\left[\sumtT \sum_{a \in \sbr*{K}}
\big(q_t(a|x)-\pi^{*}(a|x)\big)\,\langle x,\wh\theta_{t, a}\rangle\right] \\
&= \bbE\!\left[\sumtT \langle q_t(.|x)-\pi^{*}(\cdot|x),\,\wh\ell_t \spr*{x}\rangle\right].
\end{align*}

%\red{Don't be lazy! this kind of formulation is not ok in a research paper, mais c vraiment ce que absolument tous les articles font et en vrai c clair, genre c pas psq c un exo qu'on peut pas le citer nan?}
We now invoke the standard FTRL regret decomposition with time-varying regularizers (see, \eg Exercise~28.12 in \cite{BanditBook}): \DB{I think this works only with the estimate that is actually used in FTRL. So, for us, $\wt \theta_{t, a}$.}  
for any $q\in\Delta_K$,
\begin{align*}
\sumtT \langle q_t(.|x)-q,\,\wh\ell_t \spr*{x}\rangle
&\le \psi_{T+1}(q)-\psi_1(q_1)
+ \sumtT \big(\psi_t(q_{t+1}(.|x)))-\psi_{t+1}(q_{t+1}(.|x))\big) \\
&\qquad\qquad
+ \sumtT \big(\langle q_t(.|x)-q_{t+1}(.|x),\wh\ell_t \spr*{x}\rangle - D_t(q_{t+1}(.|x),q_t(.|x))\big).
\end{align*}
Choosing $q=\pi^{*}(\cdot|x)$ and taking expectations yields the claim,
with the additional additive term $\bar\beta\,\bar h$ accounting for the
regularizer variation bound used in our setup.
\end{proof}




\begin{lemma}[Lemma~15 of \cite{BOBWhardproblems}]\label{lem::ftrl_smooth_bound}
Let $l,L \in \R^K$, let $q,r \in \cP_k$ be:

\[q \in \text{arg min}_{p \in \cP_k}\{\inp*{L,p}+\beta(-H_{\alpha}(p))+\bar{\beta}(-H_{\bar{\alpha}}(p))\}\]

\[r \in \text{arg min}_{p \in \cP_k}\{\inp*{L+l,p}+\beta'(-H_{\alpha}(p))+\bar{\beta}(-H_{\bar{\alpha}}(p))\}\]

for the Tsallis entropy $H_{\alpha}$ and $0<\beta<\beta'$. Suppose also that

\[||l||_{\infty}\leq\max(\frac{1-(\sqrt{2})^{\alpha-1}}{2}q_*^{\alpha-1}\beta,\frac{1-(\sqrt{2})^{\bar{\alpha}-1}}{2}q_*^{\bar{\alpha}-1}\bar{\beta})\]

\[0 \leq \beta'-\beta \leq \max((1-(\sqrt{2})^{\alpha-1})\beta,\frac{1-(\sqrt{2})^{\bar{\alpha}-1}}{\sqrt{2}}q_*^{\bar{\alpha}-\alpha}\bar{\beta})\]

Then it holds that $H_{\alpha}(r) \leq 2H_{\alpha}(q)$.
\end{lemma}

\DB{Check carefully this thing below, cause there might be again the issue of treating some probabilities as fixed while they're not. Note that the ghost sample technique might require that we only have to control the entropy of a fixed context $X_0$, which might simplify things.}
\begin{lemma}\label{lem::cond_bound_ht}
Algorithm~\ref{alg::FTRL_bobw} satisfies, for all $t \ge 1$,
    \begin{equation*}
\bbE[h_{t+1}\mid \cH_t] \le 2\,\bbE[h_t\mid \cH_{t-1}].
    \end{equation*}
\end{lemma}

\begin{proof}
We first control the key quantities appearing in Lemma~\ref{lem::ftrl_smooth_bound}.
Recall that $\beta_t = 1/\eta_t$, $\gamma_t = \sqrt{z_t/\beta_t} + u_t/\beta_t$,
and $h_t = \frac{1}{\alpha}\sum_{i=1}^{K}(q_{t,i}^{\alpha}-q_{t,i})$.

\paragraph{Step 1: Bounding $\sqrt{z_t}$ and $h_t$.}
By definition of $z_t$ we have
    \begin{equation*}
\sqrt{z_t}
= \sqrt{\frac{4cK d^2}{1-\alpha}
\Big(\sum_{i\ne I_t}q_{t,i}^{2-\alpha}+q_{t,a_t^\star}^{2-\alpha}\Big)}
\le \frac{2d\sqrt{Kc}}{\sqrt{1-\alpha}}\,q_{t,a_t^\star}^{\,1-\frac{\alpha}{2}}.
    \end{equation*}
In addition, from the properties of the Tsallis entropy (see, \eg, Lemma 13 of \cite{BOBWhardproblems}),
    \begin{equation*}
h_t = \frac{1}{\alpha}\sum_{i=1}^{K}(q_{t,i}^{\alpha}-q_{t,i})
\ge \frac{1-(1/2)^{1-\alpha}}{\alpha}\,q_{t,a_t^\star}^{\alpha}
\ge \frac{1-\alpha}{4\alpha}\,q_{t,a_t^\star}^{\alpha}.
    \end{equation*}

\paragraph{Step 2: Bounding the variation of $\beta_t$.}
From Equation~\ref{Rule2},
    \begin{equation*}
\beta_{t+1}-\beta_t
= \frac{2}{h_t}\sqrt{\tfrac{z_t}{\beta_t}}
+ \frac{u_t}{h_t\beta_t}.
    \end{equation*}
Plugging in the bounds on $\sqrt{z_t}$ and $h_t$ gives
\begin{align*}
\beta_{t+1}-\beta_t
&\le \frac{16\alpha d\sqrt{Kc}}{\sqrt{\beta_t}\spr*{1 - \alpha}^{3/2}}\,q_{t,a_t^\star}^{\,1-\frac{3\alpha}{2}}
+ \frac{32\alpha d\max(c,1)}{\sqrt{\beta_t}\spr*{1 - \alpha}^2\lambda_{\min}}\,q_{t,a_t^\star}^{\,1-2\alpha} \\
&\le \alpha\bar{\beta}\,q_{t,a_t^\star}^{\,1-\frac{3\alpha}{2}}
+ \frac{\alpha\bar{\beta}}{\lambda_{\min}}\,q_{t,a_t^\star}^{\,1-2\alpha} \\
&\le 2\,\frac{(1-\bar{\alpha})}{\min(1,\lambda_{\min})}\,\bar{\beta}\,q_{t,a_t^\star}^{\,\bar{\alpha}-\alpha}
\le 2\,\frac{1-(\sqrt{2})^{\bar{\alpha}-1}}{\sqrt{2}}\,
\bar{\beta}\,q_{t,a_t^\star}^{\,\bar{\alpha}-\alpha}.
\end{align*}
Hence, $\beta_{t+1}-\beta_t$ satisfies the second condition of Lemma~\ref{lem::ftrl_smooth_bound}.

\paragraph{Step 3: Bounding the loss magnitude.}
For any fixed context $x$ and arm $i \in \sbr*{K}$,
\begin{align*}
|\wh{\ell}_{t+1,i} \spr*{x}|
&= |\langle x, \wh{\theta}_{t+1,i} \rangle|
\le \frac{d}{\lambda_{\min} p_t}
\le \frac{d}{\lambda_{\min}}\cdot\frac{\beta_t}{u_t} \\
&= \frac{1-\alpha}{8}\cdot\frac{\beta_t}{q_{t,a_t^\star}^{\,1-\alpha}}
\le \frac{1-(\sqrt{2})^{\alpha-1}}{2}\cdot\frac{\beta_t}{q_{t,a_t^\star}^{\,1-\alpha}}.
\end{align*}
This matches the first smoothness condition of Lemma~\ref{lem::ftrl_smooth_bound}.

\paragraph{Step 4: Applying Lemma~\ref{lem::ftrl_smooth_bound}.}
Since both smoothness conditions are satisfied, the lemma implies
    \begin{equation*}
H_{\alpha}(q_{t+1}) \le 2\,H_{\alpha}(q_t),
    \end{equation*}
and therefore $h_{t+1} \le 2h_t$ whenever the context remains fixed.

Taking conditional expectations and using the stationarity of the context distribution then yields
    \begin{equation*}
\bbE[h_{t+1}\mid \cH_t]
\le 2\,\bbE[h_t\mid \cH_{t-1}],
    \end{equation*}
which completes the proof.
\end{proof}


\begin{lemma}[Slight adaptation of Theorem~6 of \cite{BOBWhardproblems}]\label{hardthm6}

For all $\epsilon \geq 1/T$, it holds that

\begin{align*}
&\quad F\left( \beta_{1:T}, z_{1:T}, u_{1:T}, h_{0:T-1} \right) \\
&\lesssim \min \left\{ 
\left( \sumtT \sqrt{z_t h_t \log(\epsilon T)} \right)^{2/3} \right. \\
&\qquad\quad \left. ,\ \left( \frac{ \sqrt{z_{\max} h_{\max}} }{\epsilon} \right)^{2/3},
\left( \sumtT \sqrt{z_t h_{\max}} \right)^{2/3}
\right\} \\
& + \min \left\{ 
\sqrt{ \sumtT u_t h_t \log(\epsilon T) },\ 
\frac{ \sqrt{u_{\max} h_{\max}} }{\epsilon },\ 
\sumtT u_t h_{\max} 
\right\} \\
& + \sqrt{ \frac{ z_{\max} }{ \beta_1 } }
+ \frac{ u_{\max} }{ \beta_1 }
+ \beta_1 h_1
\end{align*}


\end{lemma}

\begin{proof}
This slight adaptation originates from a minor modification of Lemma~4 in \cite{BOBWhardproblems}, where in the first line of equation (24) we instead bound:
\begin{align*}
&F\left( \beta_{1:T}, z_{1:T}, u_{1:T}, h_{0:T-1} \right) 
\leq 2 \sqrt{ \frac{z_1}{\beta_1} } 
+ \frac{u_1}{\beta_1} 
+ \beta_1 h_1 \\
&\quad + \sum_{t=2}^T \left( 
2 \sqrt{ \frac{z_t}{\beta_t} } 
+ \frac{u_t}{\beta_t} 
+ (\beta_t - \beta_{t-1}) h_{t-1} 
\right).
\end{align*}


After this adjustment, the remainder of the proof proceeds identically.

\end{proof}
\clearpage
\section{MATRIX GEOMETRIC RESAMPLING}\label{sec::MGR}

Before detailing Algorithm~\ref{alg:MGR}, we elaborate on why using the parameter estimates from Eq.~\eqref{eq::estimator} \antoine{probably meant Equation~\eqref{eq::thetahat}?} would be untractable in practice. To prove this point, we detail the computation of the exact covariance matrix $\Sigma_{t, a}$, which involves evaluating the following conditional expectation
%
\begin{align*}
    \Sigma_{t, a} &= \bbE_t \sbr*{\ind_{\scbr*{a \in O_t}} X_t X_t\transpose}\\
    &= \sum_{x \in \cX} \bbP \sbr*{X_t = x, a \in O_t \given \cH_{t-1}} x x\transpose\\
    &= \sum_{x \in \cX} \bbP \sbr*{X_t = x \given \cH_{t-1}} \underbrace{\bbP \sbr*{a \in O_t \given X_t = x, \cH_{t-1}}}_{= p_t \spr*{x}} x x\transpose.
\end{align*}
%
The challenge lies in evaluating the conditional observation probability $p_t \spr*{x}$. Note that in Algorithm~\ref{alg::FTRL_bobw}, $p_t$ was defined unambiguously since it was the observation probability corresponding to the (unique) fixed context $X_t$, computed after it is revealed. Here, $p_t \spr*{x}$ is derived following the same steps, but computed as if context $x$ was observed instead of $X_t$. Doing so requires performing all computations leading to Eq.~\eqref{Rule1} separately for each possible context $x \in \cX$. This results in a computational complexity proportional to the size of the context space, $\abs*{\cX}$, which becomes quickly prohibitive when $\cX$ is large. In addition, we can note that each individual computation requires solving an optimization problem to obtain the FTRL sampling probability (Eq.~\eqref{eq::FTRL}).

To circumvent this limitation, we follow \citet{neu2013efficient, neu2016exploration, BOBWlinear} and use Matrix Geometric Resampling (MGR) to efficiently approximate the \emph{inverse} of the matrix $\Sigma_{t, a}$ directly. It does not need to compute the FTRL sampling allocation over all possible contexts but only on a carefully chosen number of sampled contexts, and only use matrix products (costing $\cO \spr*{d^2}$) but no matrix inversion (costing $\cO \spr*{d^3}$). We recall this procedure in Algorithm~\ref{alg:MGR} below. In the pseudo-code, we denote by $\cB \spr*{p}$ the Bernoulli distribution with parameter $p$.

\begin{algorithm}
    \caption{Matrix Geometric Resampling (MGR).}
    \label{alg:MGR}
    \begin{algorithmic}[1]
        \STATE {\bfseries Input:} Sampler of the context distribution $\cD$, number of iterations $M_t$.
        \STATE Initialize $\Sigma_t^+ \gets \frac{1}{2}I,\quad A_0 \gets I$.
        \FOR{$i = 1$ to $M_t$}
            \STATE Sample $X \sim \cD$.
            \STATE Compute probability of observation $p$ as in Step 5 of Algorithm~\ref{alg::FTRL_bobw} if $X_t$ was equal to $X$.
            \STATE Sample $b \sim \cB \spr*{p}$.
            \STATE Compute $B_i \gets b X X\transpose$.
            \STATE Compute $A_i \gets A_{i-1} \spr*{I - \frac12 B_i}$.
            \STATE Update $\Sigma_t^+ \gets \Sigma_t^+ + \frac12 A_i$.
        \ENDFOR
        \STATE {\bfseries Return} $\Sigma_t^+$.
    \end{algorithmic}
\end{algorithm}

We now introduce the technical results related to the cost and approximation guarantees of the MGR procedure, which will be used in the regret analysis (see the proof sketch in Section~\ref{sec::regret}).
%
\begin{lemma}[Adapted from Lemma~9 of \citealp{BOBWlinear}] \label{lem::MGRbound}
    Denote $\wh{\theta}_{t, a} = \Sigma_{t, a}^{-1} X_t \, \ell_t \spr*{X_t, a} \, \ind_{\scbr*{a \in O_t}}$ and $\wt{\theta}_{t, a} = \Sigma_{t, a}^+ X_t \, \ell_t \spr*{X_t, a} \, \ind_{\scbr*{a \in O_t}}$, where $\Sigma_{t, a}^+$ is obtained via Algorithm~\ref{alg:MGR} with the number of iterations $M_t$ tuned as in Eq.~\eqref{eq::Mt}. Then, for any arm $a \in \sbr*{K}$ and round $t \geq 1$, it holds that
    %
    \begin{equation*}
        \abs*{\bbE \sbr*{\inp*{X_t, \wt{\theta}_{t, a} - \wh{\theta}_{t, a}} \given \cH_{t-1}}} \leq \exp \spr*{- \frac{\ptmin \lambda_{\min}}{2 K} M_t}.
    \end{equation*}
\end{lemma}

\DB{$\ptmin$ can be replaced using forced exploration.}

\begin{proof}
    \antoine{Expectations should be conditional, and the estimators are incorrect (should be $\ind_{\scbr*{a \in O_t}}$ instead)} Let $\norm*{\cdot}_{\mathrm{op}}$ denote the operator norm. Denote by $\wh{\Sigma}_{t, a}^+$ the random matrix output by the MGR procedure in Algorithm~\ref{alg:MGR}. Under independence assumptions of the geometric resampling steps, we have
    %
    \begin{equation*}
        \bbE \sbr*{\prod_{j=1}^i \spr*{I - \frac12 B_j}} = \spr*{I - \frac12 \Sigma_{t, a}}^i,
    \end{equation*}
    %
    and consequently,
    %
    \begin{equation*}
        \bbE \sbr*{\wh{\Sigma}_{t, a}^+} = \frac12 \sum_{i=0}^{M_t} \spr*{I - \frac12 \Sigma_{t, a}}^i = \Sigma_{t, a}^{-1} - \spr*{I - \frac12 \Sigma_{t, a}}^{M_t} \Sigma_{t, a}^{-1}.
    \end{equation*}
    %
    Using this, we compute the expectation of the biased estimator
    %
    \begin{align*}
        \bbE \sbr*{\wt{\theta}_{t, a}} &= \bbE \sbr*{\wh{\Sigma}_{t, a}^+ X_t \, \ell_t \spr*{X_t, a} \, \ind_{\scbr*{A_t = a}}}\\
        &= \bbE \sbr*{\wh{\Sigma}_{t, a}^+} \cdot \bbE \sbr*{X_t \inp*{X_t, \theta_{t, a}} \, \ind_{\scbr*{A_t = a}}}\\
        &= \bbE \sbr*{\wh{\Sigma}_{t, a}^+} \cdot \bbE \sbr*{X_t X_t\transpose \, \ind_{\scbr*{A_t = a}}} \cdot \theta_{t, a}\\
        &= \spr*{ \Sigma_{t, a}^{-1} - \spr*{I - \frac12 \Sigma_{t, a}}^{M_t} \Sigma_{t, a}^{-1} } \cdot \Sigma_{t, a} \cdot \theta_{t, a}\\
        &= \theta_{t, a} - \spr*{I - \frac12 \Sigma_{t, a}}^{M_t} \theta_{t, a}.
    \end{align*}
    %
    Hence, the bias is given by
    %
    \begin{equation*}
        \bbE \sbr*{\wt{\theta}_{t, a} - \wh{\theta}_{t, a}} = - \spr*{I - \frac12 \Sigma_{t, a}}^{M_t} \theta_{t, a}.
    \end{equation*}
    %
    We then bound the inner product as
    %
    \begin{align*}
        \abs*{\bbE \sbr*{\inp*{X_t, \wt{\theta}_{t, a} - \wh{\theta}_{t, a}} \given \cH_{t-1}}} &\leq \norm*{X_t}_2 \, \norm*{\theta_{t, a}}_2 \, \norm*{\spr*{I - \frac12 \Sigma_{t, a}}^{M_t}}_{\mathrm{op}}\\
        &\leq \XMAX \THETAMAX \spr*{1 - \frac{\ptmin \lambda_{\min}}{2 K}}^{M_t}\\
        &\leq \exp \spr*{- \frac{\ptmin \lambda_{\min}}{2 K} M_t},
    \end{align*}
    %
    where we used $\norm*{X_t}_2 \leq \XMAX$, $\norm*{\theta_{t, a}}_2 \leq \THETAMAX$, and the bound $\Sigma_{t, a} \succeq \frac{\ptmin \lambda_{\min}}{K} I$ in the third inequality (since each arm is observed with probability $p_t$). \DB{Why do we have a $K^{-1}$ factor here?} \antoine{should not be here}
\end{proof}

\begin{lemma} \label{lem::boundBias}
    The cumulative bias introduced by the MGR approximation is uniformly bounded as
    %
    \begin{equation*}
        \sumtT \max_{a \in \sbr*{K}} \abs*{\bbE \sbr*{\inp*{X_t, \wt{\theta}_{t, a} - \wh{\theta}_{t, a}}}} \le \frac{\pi^2}{6}.
    \end{equation*}
\end{lemma}

\begin{proof}
    From Lemma~\ref{lem::MGRbound} and the definition $M_t = \left\lceil \tfrac{4 K}{\ptmin \lambda_{\min}} \log t \right\rceil$, we obtain, conditionally on $\cH_{t-1}$,
    %
    \begin{equation*}
        \abs*{\bbE \sbr*{\inp*{X_t, \wt{\theta}_{t, a} - \wh{\theta}_{t, a}} \given \cH_{t-1}}} \le \exp \spr*{- \tfrac{\ptmin \lambda_{\min}}{2 K} M_t} \le \frac{1}{t^2}.
    \end{equation*}
    %
    Taking total expectation and maximizing over $a \in \sbr*{K}$ yields
    %
    \begin{equation*}
        \max_{a \in \sbr*{K}} \abs*{\bbE \sbr*{\inp*{X_t, \wt{\theta}_{t, a} - \wh{\theta}_{t, a}}}} \le \frac{1}{t^2}.
    \end{equation*}
    %
    We finally obtain the result by summing over $t$.
\end{proof}

\clearpage
\input{sections/complexity.tex}


\end{document}

% This document was modified from the file originally made available by
% Pat Langley and Andrea Danyluk for ICML-2K. This version was created
% by Iain Murray in 2018, and modified by Alexandre Bouchard in
% 2019 and 2021 and by Csaba Szepesvari, Gang Niu and Sivan Sabato in 2022.
% Modified again in 2023 and 2024 by Sivan Sabato and Jonathan Scarlett.
% Previous contributors include Dan Roy, Lise Getoor and Tobias
% Scheffer, which was slightly modified from the 2010 version by
% Thorsten Joachims & Johannes Fuernkranz, slightly modified from the
% 2009 version by Kiri Wagstaff and Sam Roweis's 2008 version, which is
% slightly modified from Prasad Tadepalli's 2007 version which is a
% lightly changed version of the previous year's version by Andrew
% Moore, which was in turn edited from those of Kristian Kersting and
% Codrina Lauth. Alex Smola contributed to the algorithmic style files.
